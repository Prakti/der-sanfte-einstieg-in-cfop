\documentclass[11pt, a5paper]{scrbook}

\usepackage[utf8]{inputenc}
\usepackage[ngerman]{babel}

\usepackage[T1]{fontenc}
\usepackage{tgpagella}
\usepackage{tgheros}

%\usepackage[scale=0.85, a5paper]{geometry}

\usepackage{tikz}
\usepackage{rubikcube,rubikrotation,rubikpatterns}

\newcommand{\Algo}[1]{\ShowSequence{\ }{\textRubik}{#1}}
\newcommand{\Sexy}{(\Algo{R, U, Rp, Up})\ }
\newcommand{\Train}[1]{{\sffamily\small Training Scramble:} \\ {\ttfamily\footnotesize #1}}

\newenvironment{instruction}[1]{%
  \begin{minipage}{0.45\linewidth} %
    \begin{center} %
      {\sffamily\bfseries #1}\\
}{
    \end{center}
  \end{minipage}
}

\title{Der sanfte Einstieg in's Speedcubing}
\author{Marcus Autenrieth}

\begin{document}
\maketitle

Hi! Du wolltest schon immer mal einen Rubik's Würfel lösen und brauchst etwas, das Du ausdrucken und in der Tasche dabei haben kannst?
Hast Du das Ziel den Würfel in unter zwei Minuten zu lösen und suchst nach dem optimalen Einstieg um sowohl erste Erfolge zu feiern, als auch später immer schneller zu werden?
Du bist hier richtig!

\tableofcontents

% TODO: Table of Algorithms!
\chapter{Einleitung}
Diese Anleitung zielt darauf ab, es jedem Anfänger zu ermöglichen überhaupt einen Rubik's Würfel zu lösen.
Dabei wollen wir aber auch gleichzeitig die bestmöglichen Grundlagen legen, um später immer schneller darin zu werden.
Deshalb orientieren wir uns direkt an den Schritten und Zwischenergebnissen einer beliebten Speedcubing Methode: CFOP.

Damit Du Dir am Anfang nur ganz wenig merken musst, werden wir mit einer stark abgespeckten Variante dieser Methode anfangen.

Mit zunehmender Sicherheit wirst Du dann anfangen schneller zu werden und an die Grenzen der Anfängermethode stoßen.
Mit der Übung werden Dir Abläufe als Umständlich erscheinen und Du wirst Dir wünschen, mehrere Schritte in einem zu erledigen.
Hier wollen wir Dir mit diesem Leitfaden die Möglichkeite geben, Stück für Stück neue Techniken dazu zu nehmen, bis Du CFOP vollständig beherrschst.

Die CFOP-Methode basiert auf der Idee, dass sich jedes Problem einfach und wiederholbar lösen lässt, indem man es in Teilprobleme zerlegt, welche man dann einzeln löst.
Der Begriff CFOP leitet sich von den vier Teilprobleme ab, in die sich das Lösen des Würfels zerlegen lässt.
Es ist ein Akronym für Cross, F2L (First 2 Layers), OLL (Orientation of the Last Layer) und PLL (Permutation of the Last Layer):\\[1em]
\parbox{0.7\linewidth}{
\textbf{Cross (Kreuz):} Zuerst wird ein Kreuz auf einer der Würfelseiten erstellt, wobei die Kantensteine der Farbe des Kreuzes mit den Mittelsteinen der angrenzenden Seiten ausgerichtet werden.
}\parbox{0.3\linewidth}{
\centering
\RubikCubeGreyAll%
\RubikFaceUp{X}{W}{X}
            {W}{W}{W}
            {X}{W}{X}%
\RubikFaceFront{X}{O}{X}
	       {X}{O}{X}
	       {X}{X}{X}%
\RubikFaceRight{X}{G}{X}
	       {X}{G}{X}
	       {X}{X}{X}%
\ShowCube{2cm}{0.4}{\DrawRubikCube}
}\\[1em]
\parbox{0.7\linewidth}{
\textbf{F2L -- First 2 Layers (die ersten 2 Ebenen):}
In diesem Schritt werden die ersten zwei Ebenen des Würfels gleichzeitig gelöst, indem Ecken und Kanten gepaart und an ihren Platz gebracht werden.
}\parbox{0.3\linewidth}{
\centering
\RubikCubeGreyAll%
\RubikFaceUp{W}{W}{W}
            {W}{W}{W}
            {W}{W}{W}%
\RubikFaceFront{O}{O}{O}
	       {O}{O}{O}
	       {X}{X}{X}%
\RubikFaceRight{G}{G}{G}
	       {G}{G}{G}
	       {X}{X}{X}%
\ShowCube{2cm}{0.4}{\DrawRubikCube}
}\\[1em]
\parbox{0.7\linewidth}{
  \textbf{OLL -- Orientation of the Last Layer (Ausrichtung der letzten Schicht):}
  Hier werden alle Steine der letzten Ebene so gedreht, dass die Oberseite eine einheitliche Farbe hat, ohne dabei die bereits gelösten Ebenen zu stören.
}\parbox{0.3\linewidth}{
\centering
\RubikCubeGreyAll%
\RubikFaceUp{Y}{Y}{Y}
            {Y}{Y}{Y}
            {Y}{Y}{Y}%
\RubikFaceFront{X}{X}{X}
	       {G}{G}{G}
	       {G}{G}{G}%
\RubikFaceRight{X}{X}{X}
	       {O}{O}{O}
	       {O}{O}{O}%
\ShowCube{2cm}{0.4}{\DrawRubikCube}
}\\[1em]
\parbox{0.7\linewidth}{
  \textbf{PLL -- Permutation of the Last Layer (Positionierung der letzten Schicht):} 
  Im letzten Schritt werden die Steine der letzten Ebene in ihre endgültige Position gebracht, wodurch der Würfel vollständig gelöst wird.
}\parbox{0.3\linewidth}{
\centering
\RubikCubeGreyAll%
\RubikFaceUp{Y}{Y}{Y}
            {Y}{Y}{Y}
            {Y}{Y}{Y}%
\RubikFaceFront{G}{G}{G}
	       {G}{G}{G}
	       {G}{G}{G}%
\RubikFaceRight{O}{O}{O}
	       {O}{O}{O}
	       {O}{O}{O}%
\ShowCube{2cm}{0.4}{\DrawRubikCube}
}\\[1em]

Für den Anfang werden wir jedes dieser Teilprobleme noch zusätzlich zerlegen, sodass wir den Würfel in insgesamt acht Phasen lösen.
Das ermöglicht es uns, mit nur einem kleinen Satz an Algorithmen für die Lösung der acht Teilprobleme auszukommen.

Ich spreche hier immer von 'wir' obwohl ich das Buch gerade alleine schreibe.
Das liegt daran, dass ich gerade Ideen und Algorithmen von ganz vielen Menschen aufnehme und entlang meiner eigenen Lernerfahrungen aufschreibe.

Viele Personen waren an der Entwicklung von CFOP beteiligt.
Vollständig dokumentiert wurde die Methode als erstes von Anneke Treep und Kurt Dockhorn im Jahr 1981.
Weiter systematisiert und popularisiert wurde CFOP von der tschechischen Speedcuberin Jessica Fridrich.
Bis heute benutzen viele berühmte Speedcuber wie Feliks Zemdegs oder Max Park diese Methoden, bzw. eine Weiterentwicklung davon.


\chapter{CFOP für Anfänger}
Als Erstes zeigen wir Dir wie man überhaupt den Würfel lösen kann. Wir werden so viel wie möglich
mit anschaulichen Bildern erläutern um möglichen Missverständnissen, Fehlern bei der Umsetzung
und damit Frustration vorzubeugen.
\section{Begriffe}

Damit Du alle Anleitungen und Hinweise in dieser Anleitung auch gut verstehen kannst, erklären wir Dir erst einmal ein paar Begriffe.
Dazu gehört die grundlegende Struktur des Würfels, wie die Seiten in dem Würfel benannt werden und wie man die weiter unten vorgestellten Algorithmen liest.
In der Speedcubing Szene ist ein Algorithmus eine Abfolge von Drehungen um ein bestimmtes Ergebnis zu erzielen.
Algorithmen bestehen oft aus vier bis neun Zügen.
Schau gerne nach jedem der folgenden Abschnitte einmal nach, ob Du das Vorgestellte am Würfel nachvollziehen kannst.

\subsection{Steine}

Ein Rubik's Würfel wird aus drei Arten von Steinen zusammen gesetzt. \\[1em]

\parbox{0.3\linewidth}{
  \centering
  \textbf{Mittelsteine}\\[1em] \RubikCubeGreyWY\ShowCube{2cm}{0.4}{\DrawRubikCube}
}
\parbox{0.7\linewidth}{
Das besondere an den Mittelsteinen ist, dass sie \emph{niemals} ihre Position verlassen können.
Im gelösten Zustand haben alle Flächen einer Seite die gleiche Farbe wie der Mittelstein.
Jeder der Seiten dreht sich um ihren Mittelstein.
}\\[1em]
\parbox{0.3\linewidth}{
  \centering
  \textbf{Kantensteine}\\[1em]
  \RubikCubeGreyAll%
  \RubikFaceUp{X}{W}{X}
	{W}{X}{W}
	{X}{W}{X}%
  \RubikFaceFront{X}{O}{X}
	   {O}{X}{O}
	   {X}{O}{X}%
  \RubikFaceRight{X}{G}{X}
	   {G}{X}{G}
	   {X}{G}{X}%
  \ShowCube{2cm}{0.4}{\DrawRubikCube}
}
\parbox{0.7\linewidth}{
  Die Kantensteine bestehen aus 2 Farben.
}\\[1em]
\parbox{0.3\linewidth}{
  \centering
  \textbf{Ecksteine}\\[1em]
  \RubikCubeGreyAll%
  \RubikFaceUp{W}{X}{W}
	{X}{X}{X}
	{W}{X}{W}%
  \RubikFaceFront{O}{X}{O}
	   {X}{X}{X}
	   {O}{X}{O}%
  \RubikFaceRight{G}{X}{G}
	   {X}{X}{X}
	   {G}{X}{G}%
  \ShowCube{2cm}{0.4}{\DrawRubikCube}
}
\parbox{0.7\linewidth}{
  Die Ecksteine bestehen aus 3 Farben.
}

\subsection{Rotationen}
Wie bereits in der Einleitung erwähnt, nennen wir die Techniken, mit denen wir jetzt schrittweise den Würfel lösen, Algorithmen.
Zu jedem der oben erwähnten vier bis acht Teilproblemen gibt es mindestens einen Algorithmus, um es zu lösen.

Jeder Algorithmus ist im Endeffekt eine Reihe von spezifischen Drehungen, welche wir mit Buchstaben beschreiben werden.

Wir werden jetzt alle für den Anfang notwendigen Drehungen und ihre Bezeichnung einmal vorstellen.
Damit Du ein besseres Bild vor Augen hast zeigen wir Dir hier einmal den teilweise aufgefalteten Würfel aus den Beispielen:\\[1em]

\RubikCubeSolved%
\ShowCube{\linewidth}{0.4}{\DrawRubikCubeSF}\\[1em]

Wie man sieht, ist die gelbe Farbe auf der Unterseite, die blaue links und die rote Farbe hinten!

\begin{instruction}{Rechte Seite im Uhrzeigersinn}
  \vspace{1.3em}
  \RubikCubeSolved%
  \ShowCube{15mm}{0.4}{\DrawRubikCube}%
  \quad\Rubik{R}\RubikRotation{R}
  \ShowCube{15mm}{0.4}{\DrawRubikCube}%
\end{instruction}
\hfil
\begin{instruction}{Rechte Seite gegen den Uhrzeigersinn}
  \RubikCubeSolved%
  \ShowCube{15mm}{0.4}{\DrawRubikCube}%
  \quad\Rubik{Rp}\RubikRotation{Rp}
  \ShowCube{15mm}{0.4}{\DrawRubikCube}%
\end{instruction}\\[5em]
\begin{instruction}{Linke Seite im Uhrzeigersinn}
  \vspace{1.3em}
  \RubikCubeSolved%
  \ShowCube{15mm}{0.4}{\DrawRubikCube}%
  \quad\Rubik{L}\RubikRotation{L}
  \ShowCube{15mm}{0.4}{\DrawRubikCube}%
\end{instruction}
\hfil
\begin{instruction}{Linke Seite gegen den Uhrzeigersinn}
  \RubikCubeSolved%
  \ShowCube{15mm}{0.4}{\DrawRubikCube}%
  \quad\Rubik{Lp}\RubikRotation{Lp}
  \ShowCube{15mm}{0.4}{\DrawRubikCube}%
\end{instruction}
\\[5em]
\begin{instruction}{Obere Seite im Uhrzeigersinn}
  \vspace{1.3em}
  \RubikCubeSolved%
  \ShowCube{15mm}{0.4}{\DrawRubikCube}%
  \quad\Rubik{U}\RubikRotation{U}
  \ShowCube{15mm}{0.4}{\DrawRubikCube}%
\end{instruction}
\hfil
\begin{instruction}{Obere Seite gegen den Uhrzeigersinn}
  \RubikCubeSolved%
  \ShowCube{15mm}{0.4}{\DrawRubikCube}%
  \quad\Rubik{Up}\RubikRotation{Up}
  \ShowCube{15mm}{0.4}{\DrawRubikCube}%
\end{instruction}
\\[5em]
\begin{instruction}{Vorne im Uhrzeigersinn}
  \RubikCubeSolved%
  \ShowCube{15mm}{0.4}{\DrawRubikCube}%
  \quad\Rubik{F}\RubikRotation{F}
  \ShowCube{15mm}{0.4}{\DrawRubikCube}%
\end{instruction}
\hfil
\begin{instruction}{Vorne gegen den Uhrzeigersinn}
  \RubikCubeSolved%
  \ShowCube{15mm}{0.4}{\DrawRubikCube}%
  \quad\Rubik{Fp}\RubikRotation{Fp}
  \ShowCube{15mm}{0.4}{\DrawRubikCube}%
\end{instruction}
\\[5em]
\begin{instruction}{Unten im Uhrzeigersinn}
  \RubikCubeSolved%
  \ShowCube{15mm}{0.4}{\DrawRubikCube}%
  \quad\Rubik{D}\RubikRotation{D}
  \ShowCube{15mm}{0.4}{\DrawRubikCube}%
\end{instruction}
\hfil
\begin{instruction}{Unten gegen den Uhrzeigersinn}
  \RubikCubeSolved%
  \ShowCube{15mm}{0.4}{\DrawRubikCube}%
  \quad\Rubik{Dp}\RubikRotation{Dp}
  \ShowCube{15mm}{0.4}{\DrawRubikCube}%
\end{instruction}
\\[2em]

Nimm Dir jetzt auch ruhig einmal Deinen Würfel zur Hand und probier ein wenig aus.
Das hilft Dir ein Gefühl für die Drehungen zu entwickeln und Dich später nicht zu verhaspeln.

\section{Überblick über die Anfängermethode}
Um es zu Anfang einfacher zu haben, zerlegen wir die vier Phasen von CFOP jeweils nochmal in zwei Teilphasen.
Wir stellen Dir jetzt diese acht Teilschritte samt Ergebnist einmal kurz vor.

\emph{Ein Hinweis dabei:} bei Flächen die auf unseren Beispielen grau gefärbt sind ist uns die Farbe in dem Zwischenschritt egal.
Das ermöglichet es uns, Dir die wesentlichen Aspekte der jeweiligen Phase zu zeigen.

\paragraph{Cross (Kreuz):} \hfill \\[1em]
\parbox{0.7\linewidth}{
\textbf{Das Gänseblümchen:} Mit diesem Zwischenschritt macht man das Bilden des Kreuzes einfacher.
}\parbox{0.3\linewidth}{
\centering
\RubikCubeGreyAll%
\RubikFaceUp{X}{W}{X}
            {W}{Y}{W}
            {X}{W}{X}%
\RubikFaceFront{X}{X}{X}
	       {X}{O}{X}
	       {X}{X}{X}%
\RubikFaceRight{X}{X}{X}
	       {X}{G}{X}
	       {X}{X}{X}%
\ShowCube{3cm}{0.4}{\DrawRubikCube}
}\\[1em]
\parbox{0.7\linewidth}{
\textbf{Das weiße Kreuz:} Jetzt machen wir daraus das Kreuz auf der weißen Seite, wobei die Kantensteine der Farbe des Kreuzes mit den Mittelsteinen der angrenzenden Seiten ausgerichtet werden.
}\parbox{0.3\linewidth}{
\centering
\RubikCubeGreyAll%
\RubikFaceUp{X}{W}{X}
            {W}{W}{W}
            {X}{W}{X}%
\RubikFaceFront{X}{O}{X}
	       {X}{O}{X}
	       {X}{X}{X}%
\RubikFaceRight{X}{G}{X}
	       {X}{G}{X}
	       {X}{X}{X}%
\ShowCube{3cm}{0.4}{\DrawRubikCube}
}\\[1em]
\paragraph{F2L -- First 2 Layers (die ersten 2 Ebenen):}\hfill\\[1em]
\parbox{0.7\linewidth}{
\textbf{Die erste Ebene fertig stellen:} Nachdem das Kreuz fertig ist bringen wir die Ecksteine an ihre korrekte Stelle und richten sie aus.
}\parbox{0.3\linewidth}{
\centering
\RubikCubeGreyAll%
\RubikFaceUp{W}{W}{W}
            {W}{W}{W}
            {W}{W}{W}%
\RubikFaceFront{O}{O}{O}
	       {X}{O}{X}
	       {X}{X}{X}%
\RubikFaceRight{G}{G}{G}
	       {X}{G}{X}
	       {X}{X}{X}%
\ShowCube{3cm}{0.4}{\DrawRubikCube}
}\\[1em]
\parbox{0.7\linewidth}{
\textbf{Die zweite Ebene fertig stellen:} Jetzt lösen wir die zweite Ebene indem wir die passenden Kantensteine an ihren Platz bringen.
}\parbox{0.3\linewidth}{
\centering
\RubikCubeGreyAll%
\RubikFaceUp{W}{W}{W}
            {W}{W}{W}
            {W}{W}{W}%
\RubikFaceFront{O}{O}{O}
	       {O}{O}{O}
	       {X}{X}{X}%
\RubikFaceRight{G}{G}{G}
	       {G}{G}{G}
	       {X}{X}{X}%
\ShowCube{3cm}{0.4}{\DrawRubikCube}
}\\[1em]

\paragraph{OLL -- Orientation of the Last Layer (Ausrichtung der letzten Schicht):}\hfill\\[1em]
\parbox{0.7\linewidth}{
  \textbf{Das gelbe Kreuz:} Als Erstes erzeugen wir ein gelbes Kreuz in der letzten Ebene
}\parbox{0.3\linewidth}{
\centering
\RubikCubeGreyAll%
\RubikFaceUp{X}{Y}{X}
            {Y}{Y}{Y}
            {X}{Y}{X}%
\RubikFaceFront{X}{X}{X}
	       {G}{G}{G}
	       {G}{G}{G}%
\RubikFaceRight{X}{X}{X}
	       {O}{O}{O}
	       {O}{O}{O}%
\ShowCube{3cm}{0.4}{\DrawRubikCube}
}\\[1em]
\parbox{0.7\linewidth}{
  \textbf{Die gelbe Fläche:} Dann Vervollständigen wir die gelbe Fläche.
}\parbox{0.3\linewidth}{
\centering
\RubikCubeGreyAll%
\RubikFaceUp{Y}{Y}{Y}
            {Y}{Y}{Y}
            {Y}{Y}{Y}%
\RubikFaceFront{X}{X}{X}
	       {G}{G}{G}
	       {G}{G}{G}%
\RubikFaceRight{X}{X}{X}
	       {O}{O}{O}
	       {O}{O}{O}%
\ShowCube{3cm}{0.4}{\DrawRubikCube}
}\\[1em]
\paragraph{PLL -- Permutation of the Last Layer (Positionierung der letzten Schicht):}\hfill\\[1em]
\parbox{0.7\linewidth}{
  \textbf{Die Ecksteine an ihre Position bringen:} Hier sorgen wir erst für die korrekte Positionierung der Ecksteine der letzten Ebene.
}\parbox{0.3\linewidth}{
\centering
\RubikCubeGreyAll%
\RubikFaceUp{Y}{Y}{Y}
            {Y}{Y}{Y}
            {Y}{Y}{Y}%
\RubikFaceFront{G}{X}{G}
	       {G}{G}{G}
	       {G}{G}{G}%
\RubikFaceRight{O}{X}{O}
	       {O}{O}{O}
	       {O}{O}{O}%
\ShowCube{3cm}{0.4}{\DrawRubikCube}
}\\[1em]
\parbox{0.7\linewidth}{
  \textbf{Die Kantensteine an ihre Position bringen:} Dann sorgen wir erst für die korrekte Positionierung der Kantensteine der letzten Ebene und damit ist der Würfel dann fertig gelöst.
}\parbox{0.3\linewidth}{
\centering
\RubikCubeGreyAll%
\RubikFaceUp{Y}{Y}{Y}
            {Y}{Y}{Y}
            {Y}{Y}{Y}%
\RubikFaceFront{G}{G}{G}
	       {G}{G}{G}
	       {G}{G}{G}%
\RubikFaceRight{O}{O}{O}
	       {O}{O}{O}
	       {O}{O}{O}%
\ShowCube{3cm}{0.4}{\DrawRubikCube}
}\\[1em]


Jetzt da Du einen groben Überblick hast, erläutern wir Dir jede dieser acht Phasen im Detail:

\section{Das Gänseblümchen}

\parbox{0.7\linewidth}{
  Um einen Rubik's Cube zu lösen, muss im ersten Schritt ein Gänseblümchen auf der Seite mit dem gelben Mittelstein gebildet werden.
  Dein Ziel ist es, die vier weißen Kantensteine nach oben zu verschieben, sodass sie den gelben Mittelstein umgeben, wie rechts abgebildet gezeigt:\\[1em]
} \parbox{0.3\linewidth}{
  \RubikCubeGreyAll%
  \RubikFaceUp{X}{W}{X}
	      {W}{Y}{W}
	      {X}{W}{X}%
  \RubikFaceFront{X}{X}{X}
		 {X}{R}{X}
		 {X}{X}{X}%
  \RubikFaceRight{X}{X}{X}
		 {X}{G}{X}
		 {X}{X}{X}%
  \ShowCube{3cm}{0.4}{\DrawRubikCube}\\[1em]
}

Die ersten drei Kantensteine sind relativ einfach nach oben zu bekommen.
Hier musst Du einfach etwas experimentieren.
Das schaffst Du, da sind wir uns ganz sicher!

Der letzte Kantenstein kann jedoch etwas kniffliger sein.
Im Folgenden zeigen wir dir drei mögliche Fälle und wie man sie löst.
\emph{Bedenke bitte:} es sind nur Lösungsvorschläge.
Da noch kaum etwas vom Würfel gelöst ist, kann man nicht viel kaputt machen.
Entsprechend gibt es viel Spielraum für das verschrieben von Steinen.

\paragraph{Fall 1}
\RubikCubeGreyAll%
\RubikFaceUp{X}{W}{X}
            {W}{Y}{X}
            {X}{W}{X}%
\RubikFaceFront{X}{X}{X}
               {X}{R}{W}
	       {X}{X}{X}%
\RubikFaceRight{X}{X}{X}
	       {X}{G}{X}
	       {X}{X}{X}%
\ShowCube{2cm}{0.4}{\DrawRubikCube}
\quad\Rubik{R}\RubikRotation{R}
\ShowCube{2cm}{0.4}{\DrawRubikCube}

\paragraph{Fall 2}
\RubikCubeGreyAll%
\RubikFaceUp{X}{W}{X}
            {W}{Y}{W}
            {X}{X}{X}%
\RubikFaceFront{X}{X}{X}
               {X}{R}{W}
	       {X}{X}{X}%
\RubikFaceRight{X}{X}{X}
	       {X}{G}{X}
	       {X}{X}{X}%
\ShowCube{2cm}{0.4}{\DrawRubikCube}
\quad\Rubik{Up}\RubikRotation{Up}
\ShowCube{2cm}{0.4}{\DrawRubikCube}
\quad\Rubik{R}\RubikRotation{R}
\ShowCube{2cm}{0.4}{\DrawRubikCube}

\paragraph{Fall 3}
\RubikCubeGreyAll%
\RubikFaceUp{X}{W}{X}
            {W}{Y}{W}
            {X}{X}{X}%
\RubikFaceFront{X}{W}{X}
               {X}{R}{X}
	       {X}{X}{X}%
\RubikFaceRight{X}{X}{X}
	       {X}{G}{X}
	       {X}{X}{X}%
\ShowCube{2cm}{0.4}{\DrawRubikCube}
\quad\Rubik{F}\RubikRotation{F}
\ShowCube{2cm}{0.4}{\DrawRubikCube}
\quad\Rubik{Up}\RubikRotation{Up}
\ShowCube{2cm}{0.4}{\DrawRubikCube}
\quad\Rubik{R}\RubikRotation{R}
\ShowCube{2cm}{0.4}{\DrawRubikCube}

\section{Das weiße Kreuz}
\parbox{0.7\linewidth}{
In nächsten Schritt soll ein weißes Kreuz gebildet werden, wie auf dem folgenden Bild zu sehen ist.
Um ein weißes Kreuz zu bilden, müssen die vier weißen Kantensteine um den weißen Mittelstein platziert werden.
}\parbox{0.3\linewidth}{
  \RubikCubeGreyAll%
  \RubikFaceUp{X}{W}{X}
	      {W}{W}{W}
	      {X}{W}{X}%
  \RubikFaceFront{X}{R}{X}
		 {X}{R}{X}
		 {X}{X}{X}%
  \RubikFaceRight{X}{B}{X}
		 {X}{B}{X}
		 {X}{X}{X}%
  \ShowCube{5cm}{0.4}{\DrawRubikCube}
}\\[1em]

Wie in der Illustration eben gezeigt, geht es dabei auch darum, dass die weißen Kantensteine des Kreuzes zu den Mittelsteinen der vier angrenzenden Flächen passen.
Also weiß-rot zu rot, weiß-blau zu blau, weiß-grün zu grün, weiß-orange zu orange.

Dies bewerkstelligen wir jetzt, indem wir die Oberseite des Würfels drehen, bis der weiße Kantenstein mit seiner anderen Farbe zum angrenzenden Mittelstein einer Seite passt.
Dann drehen wir diese Seite um 180°.
Im folgenden Beispiel zeigen wir das einmal für die grüne Seite. Erst drehen wir den grün-weißen Kantenstein über den grünen Eckstein, und dann drehen wir ihn "'runter"' auf die weiße Ebene:\\[1em]

\RubikCubeGreyAll%
\RubikFaceUp{X}{W}{X}
            {W}{Y}{W}
            {X}{W}{X}%
\RubikFaceFront{X}{G}{X}
               {X}{R}{X}
	       {X}{X}{X}%
\RubikFaceRight{X}{B}{X}
	       {X}{G}{X}
	       {X}{X}{X}%
\RubikFaceLeft{X}{O}{X}
	       {X}{B}{X}
	       {X}{X}{X}%
\ShowCube{2cm}{0.4}{\DrawRubikCube}
\quad\Rubik{Up}\RubikRotation{Up}
\ShowCube{2cm}{0.4}{\DrawRubikCube}
\quad\Rubik{R}\RubikRotation{R}
\ShowCube{2cm}{0.4}{\DrawRubikCube}
\quad\Rubik{R}\RubikRotation{R}
\ShowCube{2cm}{0.4}{\DrawRubikCube}\\[1em]

Damit hast Du den ersten Stein im weißen Kreuz gelöst.
Wiederhole dies für die anderen 3 Steine, um das komplette weiße Kreuz auf dem Würfel zu erzielen.
Am Ende soll der Würfel dann so aussehen:\\[1em]
\RubikCubeGreyAll%
\RubikFaceUp{X}{X}{X}
            {X}{Y}{X}
            {X}{X}{X}%
\RubikFaceFront{X}{X}{X}
               {X}{R}{X}
	       {X}{R}{X}%
\RubikFaceRight{X}{X}{X}
	       {X}{G}{X}
	       {X}{G}{X}%
\ShowCube{5cm}{0.4}{\DrawRubikCube}\\[1em]

Jetzt haben wir das "'C"' der CFOP Methode abgehakt und kümmern uns nacheinander um die
Fertigstellung der ersten und danach der zweiten Schicht.

\section{Die weiße Ebene vervollständigen}
\parbox{0.7\linewidth}{
Wir vervollständigen jetzt die weiße Ebene, indem wir die Ecksteine um das Kreuz herum an ihre korrekte Position bringen.
Dabei werden wir auch darauf achten, das die Ecksteine richtig ausgerichtet sind.
Es soll nach diesem Schritt wie rechts abgebildet aussehen.

\emph{Aber:} Wir werden die weiße Ebene "'auf den Kopf gestellt"' lösen.
Das heißt: Du musst den Würfel so halten, dass die gelbe Seite oben ist.
}\parbox{0.3\linewidth}{
\RubikCubeGreyAll%
\RubikFaceUpAll{W}%
\RubikFaceFront{R}{R}{R}
               {X}{R}{X}
	       {X}{X}{X}%
\RubikFaceRight{B}{B}{B}
	       {X}{B}{X}
	       {X}{X}{X}%
\ShowCube{\linewidth}{0.4}{\DrawRubikCube}
}\\[1em]
Dieses "'auf dem Kopf gestellt lösen"' fühlt sich wahrscheinlich erst einmal umständlich an.
Aber probier es bitte aus.
Du wirst feststellen, dass Du auch durch ein leichtes Kippen des Würfels überprüfen kannst, ob Du gerade die weiße Ebene korrekt löst.

Zudem hat diese Vorgehensweise mehrere Vorteile:
Sie ist der Einstieg in eine sehr geläufige Bewegung, die in viele Algorithmen vorkommt.
Und sie trainiert jetzt schon Dein intuitives Gefühl, um später die ersten zwei Ebenen in einer Phase zu lösen.
Und dadurch dass Du jetzt alle Phasen mit der gelben Ebene nach oben ausführst, sparst Du Dir auch das Umdrehen des Würfels.
Das ist jetzt vielleicht nicht so gravierend, spart Dir später aber ein bis zwei wertvolle Sekunden.\\[1em]
\parbox{0.7\linewidth}{
Mit der gelben Seite nach oben, soll der Würfel nach dieser Phase dann so aussehen:

\emph{Hinweis:} die Balken unter dem Würfel zeigen die Farbe der verdeckten unteren Seite der vorderen Steine.
}\parbox{0.3\linewidth}{
\RubikCubeGreyAll%
\RubikFaceUp{X}{X}{X}
            {X}{Y}{X}
            {X}{X}{X}%
\RubikFaceFront{X}{X}{X}
               {X}{R}{X}
	       {R}{R}{R}%
\RubikFaceRight{X}{X}{X}
	       {X}{G}{X}
	       {G}{G}{G}%
\RubikFaceDown{W}{W}{W}
	      {W}{W}{W}
	      {W}{W}{W}%
\ShowCube{\linewidth}{0.4}{\DrawRubikCubeRU \DrawRubikCubeSidebarFD{RU}}
}

\pagebreak
Um die Platzierung der Ecksteine zu bestimmen, schaue dir die zwei Mittelsteine um die zu lösende Ecke an:\\[1em]
\RubikCubeGreyAll%
\RubikFaceUp{X}{X}{X}
            {X}{Y}{X}
            {X}{X}{X}%
\RubikFaceFront{X}{X}{X}
               {X}{R}{X}
	       {X}{R}{X}%
\RubikFaceRight{X}{X}{X}
	       {X}{G}{X}
	       {X}{G}{X}%
\RubikFaceDown{X}{W}{X}
	      {W}{W}{W}
	      {X}{W}{X}%
\ShowCube{\linewidth}{0.4}{\DrawRubikCubeRU \DrawRubikCubeSidebarFD{RU}}\\[1em]
In diesem Beispiel werden die Ecksteine die Farben Rot, Grün und Weiß haben.

Bringe die Ecke über die Position, an die der Stein gelangen soll. Dazu drehst
Du einfach die oberste (gelbe) Ebene. Es sollte dann so aussehen wie in einem der 3
folgend dargestellten Fälle: \\[1em]

\RubikCubeGreyAll%
\RubikFaceUp{X}{X}{X}
            {X}{Y}{X}
            {X}{X}{G}%
\RubikFaceFront{X}{X}{R}
               {X}{R}{X}
	       {X}{R}{X}%
\RubikFaceRight{W}{X}{X}
	       {X}{G}{X}
	       {X}{G}{X}%
\RubikFaceDown{X}{W}{X}
	      {W}{W}{W}
	      {X}{W}{X}%
\ShowCube{0.3\linewidth}{0.4}{\DrawRubikCubeRU \DrawRubikCubeSidebarFD{RU}}
%
\RubikCubeGreyAll%
\RubikFaceUp{X}{X}{X}
            {X}{Y}{X}
            {X}{X}{W}%
\RubikFaceFront{X}{X}{G}
               {X}{R}{X}
	       {X}{R}{X}%
\RubikFaceRight{R}{X}{X}
	       {X}{G}{X}
	       {X}{G}{X}%
\RubikFaceDown{X}{W}{X}
	      {W}{W}{W}
	      {X}{W}{X}%
\ShowCube{0.3\linewidth}{0.4}{\DrawRubikCubeRU \DrawRubikCubeSidebarFD{RU}}
%
\RubikCubeGreyAll%
\RubikFaceUp{X}{X}{X}
            {X}{Y}{X}
            {X}{X}{R}%
\RubikFaceFront{X}{X}{W}
               {X}{R}{X}
	       {X}{R}{X}%
\RubikFaceRight{G}{X}{X}
	       {X}{G}{X}
	       {X}{G}{X}%
\RubikFaceDown{X}{W}{X}
	      {W}{W}{W}
	      {X}{W}{X}%
\ShowCube{0.3\linewidth}{0.4}{\DrawRubikCubeRU \DrawRubikCubeSidebarFD{RU}}
\\[3em]
Um diese Fälle zu lösen widerhole den Algorithmus
\begin{center}
  \Algo{R, U, Rp, Up}
\end{center}
bis die Ecke an der richtigen Stelle ist und die Farben auch richtig
ausgerichtet sind.

Gehe dann zu nächsten Ecke des weißen Kreuzes und bringe durch Drehen der
obersten Ebene wieder eine passende Ecke darüber und wiederhole den Algorithmus.

\pagebreak
\subsection{Ein Sonderfall}
Es kann vorkommen, dass sich die zu lösende Ecke bereits auf der untersten Ebene befindet, aber nicht korrekt ausgerichtet (d.h. verdreht) ist.
Auch in diesem Fall wenden wir den oben beschriebenen Algorithmus an und wiederholen ihn, bis die Ecke korrekt gelöst ist:
\begin{center}
  \Algo{R, U, Rp, Up}
\end{center}
Hier zwei Beispiele für diesen Sonderfall:\\[1em]
\RubikCubeGreyAll%
\RubikFaceUp{X}{X}{X}
            {X}{Y}{X}
            {X}{X}{X}%
\RubikFaceFront{X}{X}{X}
               {X}{R}{X}
	       {X}{R}{W}%
\RubikFaceRight{X}{X}{X}
	       {X}{G}{X}
	       {R}{G}{X}%
\RubikFaceDown{X}{W}{G}
	      {W}{W}{W}
	      {X}{W}{X}%
\ShowCube{0.5\linewidth}{0.4}{\DrawRubikCubeRU \DrawRubikCubeSidebarFD{RU}}
%
\RubikCubeGreyAll%
\RubikFaceUp{X}{X}{X}
            {X}{Y}{X}
            {X}{X}{X}%
\RubikFaceFront{X}{X}{X}
               {X}{R}{X}
	       {X}{R}{G}%
\RubikFaceRight{X}{X}{X}
	       {X}{G}{X}
	       {W}{G}{X}%
\RubikFaceDown{X}{W}{R}
	      {W}{W}{W}
	      {X}{W}{X}%
\ShowCube{0.5\linewidth}{0.4}{\DrawRubikCubeRU \DrawRubikCubeSidebarFD{RU}}


\subsection{Noch ein Sonderfall}
Es kann vorkommen, dass eine Ecke sich auf der untersten (weißen) Ebene, aber an der falschen Ecke befindet:\\[1em]
\RubikCubeGreyAll%
\RubikFaceUp{X}{X}{X}
            {X}{Y}{X}
            {X}{X}{X}%
\RubikFaceFront{X}{X}{X}
               {X}{B}{X}
	       {X}{B}{G}%
\RubikFaceRight{X}{X}{X}
	       {X}{R}{X}
	       {W}{R}{X}%
\RubikFaceDown{X}{W}{R}
	      {W}{W}{W}
	      {X}{W}{X}%
\ShowCube{\linewidth}{0.4}{\DrawRubikCubeRU \DrawRubikCubeSidebarFD{RU}}

Dieses Problem lösen wir wieder mit dem oben bereits beschriebenen Algorithmus:
\begin{center}
\Algo{R, U, Rp, Up}
\end{center}
Diesmal wenden wir ihn einmal an um den Ecksteine auf die oberste Ebezu bekommen.
Dann verschieben wir ihn über seine Ziel-Ecke und wenden wieder den gewohnten Algorithmus an um ihn dort korrekt zu platzieren.
Dort kannst Du die Ecke dann wie zuvor beschrieben über ihre Ziel-Ecke bringen und auch mit

\pagebreak

\section{Die zweite Ebene fertig stellen}
\parbox{0.7\linewidth}{
In diesem Schritt geht es darum, die richtigen Kantensteine in die mittlere Ebene zu bekommen.
Nach diesem Schritt hast Du die beiden ersten Ebenen von des Würfels gelöst.
Das sieht dann so aus wie rechts abgebildet:
}
\RubikCubeGreyAll%
\RubikFaceUp{X}{X}{X}
            {X}{Y}{X}
            {X}{X}{X}%
\RubikFaceFront{X}{X}{X}
               {R}{R}{R}
	       {R}{R}{R}%
\RubikFaceRight{X}{X}{X}
	       {G}{G}{G}
	       {G}{G}{G}%
\ShowCube{0.3\linewidth}{0.4}{\DrawRubikCubeRU}\\[1em]

Halte den Würfel so, dass die weiße Seite unten ist.
Suche dann einen Kantenstein auf der oberen Ebene, der keine gelbe Seite hat.
In diesem Beispiel haben wir einen Kantenstein mit den Farben Rot und Grün gefunden.

Wenn Du einen Kantenstein gefunden hast, der nicht gelb ist, drehe die obere Ebene so lange, bis du auf einer Seite des Würfels ein umgedrehtes „T" siehst.
Auf dem Bild oben ist es ein rotes T, aber es hätte auch ein blaues, orangefarbenes oder grünes sein können. 
Die Seite mit dem T sollte Dir zugewandt sein.

\subsection{Bewegen nach Rechts}
\RubikCubeGreyAll%
\RubikFaceUp{X}{X}{X}
            {X}{Y}{X}
            {X}{G}{X}%
\RubikFaceFront{X}{R}{X}
               {X}{R}{X}
	       {R}{R}{R}%
\RubikFaceRight{X}{X}{X}
	       {X}{G}{X}
	       {G}{G}{G}%
\ShowCube{0.3\linewidth}{0.4}{%
  \DrawRubikCubeRU%
  \draw[thick,->,color=black] (1.5,2.5) -- (2.5, 1.5);
}
\parbox{0.7\linewidth}{
  In diesem Fall soll die Kante nach \textbf{rechts} bewegt werden, da die andere
  Farbe des Kantensteins grün ist. Um den Kantenstein rechts von der mittleren
  Ebene nach unten zu bekommen, führe den folgenden Algorithmus aus:
}
\begin{center}
\Algo{U, R, Up, Rp, Up, Fp, U, F}.
\end{center}


\subsection{Bewegen nach Links}
Nachdem du ein T gebildet hast, wirst du feststellen, dass in einigen Fällen
das zu lösende Kantenstück auf die linke Seite muss, wie in der Abbildung oben
gezeigt.\\[1em]

\RubikCubeGreyAll%
\RubikFaceUp{X}{X}{X}
            {X}{Y}{X}
            {X}{B}{X}%
\RubikFaceFront{X}{R}{X}
               {X}{R}{X}
	       {R}{R}{R}%
\RubikFaceRight{X}{X}{X}
	       {X}{G}{X}
	       {G}{G}{G}%
\ShowCube{0.3\linewidth}{0.4}{%
  \DrawRubikCubeRU%
  \draw[thick,->,color=black] (1.5,2.5) -- (0.5, 1.5);
}
\parbox{0.7\linewidth}{
  In diesem Beispiel soll ein Kantenstein mit den Farben Rot und Blau links von
  der mittleren Ebene platziert werden. Um dieses Problem zu lösen, führe den
  Algorithmus in umgekehrter Reihenfolge aus, wie wir ihn in diesem Schritt
  vorgestellt haben. Der Algorithmus sieht dann folgendermaßen aus:
}
\begin{center}
\Algo{Up, Lp, U, L, U, F, Up, Fp}
\end{center}

\subsection{Sonderfälle}

\parbox{0.7\linewidth}{
  \paragraph{Situation 1}
  Der Kantenstein ist an der richtigen Position aber spiegelverkert ausgerichtet.\\[1em]
}
\RubikCubeGreyAll%
\RubikFaceUp{X}{X}{X}
            {X}{Y}{X}
            {X}{X}{X}%
\RubikFaceFront{X}{X}{X}
               {X}{R}{G}
	       {R}{R}{R}%
\RubikFaceRight{X}{X}{X}
	       {R}{G}{X}
	       {G}{G}{G}%
\ShowCube{0.3\linewidth}{0.4}{%
  \DrawRubikCubeRU%
}\\[1em]
%%
\parbox{0.7\linewidth}{
  \paragraph{Situation 2}
  Der Kantenstein ist an der falschen Stelle innerhalb der mittleren Ebene.\\[1em]
}
\RubikCubeGreyAll%
\RubikFaceUp{X}{X}{X}
            {X}{Y}{X}
            {X}{X}{X}%
\RubikFaceFront{X}{X}{X}
               {X}{O}{G}
	       {O}{O}{O}%
\RubikFaceRight{X}{X}{X}
	       {R}{G}{X}
	       {G}{G}{G}%
\ShowCube{0.3\linewidth}{0.4}{%
  \DrawRubikCubeRU%
}\\[1em]

Um diese Sonderfälle zu lösen, suche Dir zunächst einen Kantenstein mit einer
gelben Seite und bringe ihn so in Position als würdest Du ihn gerne nach links
oder nach rechts an die Stelle bringen wo jetzt der unpassende Kantenstein ist.
Führe den passenden Algorithmus aus, um den gelben Kantenstein an diese Position
zu bringen. Damit ist dann der andere Stein in der oberen Ebene und kann wie
oben dargestellt mit einem der beiden Algorithmen eingebaut werden.

Jetzt haben wir die ersten zwei Ebenen fertig und damit das "'F"' der CFOP Methode abgehakt.

\section{Das gelbe Kreuz}
\parbox{0.7\linewidth}{
  Als Nächstes geht es uns darum, die letzte, gelbe Ebene so auszurichten, dass alle gelben Flächen in die richtige Richtung zeigen. 
  Wie angekündigt machen wir das in zwei Schritten.
  Der erste Schritt hat das Ziel, ein gelbes Kreuz zu bilden.
  Mit anderen Worten: Alle vier gelben Kantensteine sollen mit ihrer gelben Seite nach oben zeigen.
}
\RubikCubeGreyAll%
\RubikFaceUp{X}{Y}{X}
            {Y}{Y}{Y}
            {X}{Y}{X}%
\RubikFaceFront{X}{X}{X}
               {R}{R}{R}
	       {R}{R}{R}%
\RubikFaceRight{X}{X}{X}
	       {G}{G}{G}
	       {G}{G}{G}%
\ShowCube{0.3\linewidth}{0.4}{%
  \DrawRubikCubeRU%
}\\[1em]

Sobald die ersten zwei Ebenen des Würfels gelöst sind, kann man auf einen der vier folgenden Fälle stoßen:

\RubikCubeGreyAll%
\RubikFaceUp{X}{X}{X}
            {X}{Y}{X}
            {X}{X}{X}%
\RubikFaceFront{X}{Y}{X}
               {R}{R}{R}
	       {R}{R}{R}%
\RubikFaceRight{X}{Y}{X}
	       {G}{G}{G}
	       {G}{G}{G}%
\ShowCube{2.5cm}{0.4}{%
  \DrawRubikCubeRU%
}
$\Longrightarrow$
\RubikCubeGreyAll%
\RubikFaceUp{X}{Y}{X}
            {Y}{Y}{X}
            {X}{X}{X}%
\RubikFaceFront{X}{Y}{X}
               {R}{R}{R}
	       {R}{R}{R}%
\RubikFaceRight{X}{Y}{X}
	       {G}{G}{G}
	       {G}{G}{G}%
\ShowCube{2.5cm}{0.4}{%
  \DrawRubikCubeRU%
}
$\Longrightarrow$
\RubikCubeGreyAll%
\RubikFaceUp{X}{X}{X}
            {Y}{Y}{Y}
            {X}{X}{X}%
\RubikFaceFront{X}{Y}{X}
               {R}{R}{R}
	       {R}{R}{R}%
\RubikFaceRight{X}{X}{X}
	       {G}{G}{G}
	       {G}{G}{G}%
\ShowCube{2.5cm}{0.4}{%
  \DrawRubikCubeRU%
}
$\Longrightarrow$
\RubikCubeGreyAll%
\RubikFaceUp{X}{Y}{X}
            {Y}{Y}{Y}
            {X}{Y}{X}%
\RubikFaceFront{X}{X}{X}
               {R}{R}{R}
	       {R}{R}{R}%
\RubikFaceRight{X}{X}{X}
	       {G}{G}{G}
	       {G}{G}{G}%
\ShowCube{2.5cm}{0.4}{%
  \DrawRubikCubeRU%
}\\[1em]

Der vierte Fall ist das gesuchte gelbe Kreuz.
Manchmal ist man ohne etwas tun zu müssen in dieser Situation und ist direkt fertig.
Für die anderen drei Fälle wendest Du den Algorithmus \\[1em]
\begin{center}
  \Algo{F, R, U, Rp, Up, Fp}
\end{center}
an.

Wie durch die Pfeile in der Abbildung gezeigt bringt dies den Würfel von einer der Situationen zur nächsten.
Zwischendurch musst Du den Würfel immer wieder so drehen, dass er passend zu Dir ausgerichtet ist. 
Die obige Abbildung zeigt Dir die korrekte Ausrichtung für jeden der drei Fälle.

\selectlanguage{ngerman}
\section{Orientierung der gelben Ecken}
\parbox{0.7\linewidth}{
In diesem Schritt komplettieren wir das "'O"' von CFOP indem wir die Ecken um das Kreuz herum so orientieren, sodass sich eine gelbe Fläche ergibt.
Ob die anderen Farben stimmen ist jetzt noch nicht wichtig.
Darum kümmern wir uns später.

Folgende Fälle sind möglich:
}
\RubikCubeGreyAll%
\RubikFaceUp{Y}{Y}{Y}
            {Y}{Y}{Y}
            {Y}{Y}{Y}%
\RubikFaceFront{X}{X}{X}
               {R}{R}{R}
	       {R}{R}{R}%
\RubikFaceRight{X}{X}{X}
	       {G}{G}{G}
	       {G}{G}{G}%
\RubikFaceLeft{X}{X}{X}
	      {B}{B}{B}
	      {B}{B}{B}%
\RubikFaceBack{X}{X}{X}
	      {O}{O}{O}
	      {O}{O}{O}%
\ShowCube{0.3\linewidth}{0.4}{%
  \DrawRubikCubeRU
}\\[1em]

\paragraph{Fall 1}
\RubikCubeGreyAll%
\RubikFaceUp{Y}{Y}{Y}
            {Y}{Y}{Y}
            {Y}{Y}{Y}%
\RubikFaceFront{X}{X}{X}
               {R}{R}{R}
	       {R}{R}{R}%
\RubikFaceRight{X}{X}{X}
	       {G}{G}{G}
	       {G}{G}{G}%
\RubikFaceLeft{X}{X}{X}
	      {B}{B}{B}
	      {B}{B}{B}%
\RubikFaceBack{X}{X}{X}
	      {O}{O}{O}
	      {O}{O}{O}%
\ShowCube{2cm}{0.4}{\DrawRubikCubeRU}
\ShowCube{2.5cm}{0.4}{%
  \DrawRubikFaceUpSide
}\\[1em]
Alle Ecken sind bereits richtig orientiert. Hier sind wir direkt fertig.

\paragraph{Fall 2}
\RubikCubeGreyAll%
\RubikFaceUp{X}{Y}{X}
            {Y}{Y}{Y}
            {Y}{Y}{X}%
\RubikFaceFront{X}{X}{X}
               {R}{R}{R}
	       {R}{R}{R}%
\RubikFaceRight{X}{X}{X}
	       {G}{G}{G}
	       {G}{G}{G}%
\RubikFaceLeft{X}{X}{X}
	      {B}{B}{B}
	      {B}{B}{B}%
\RubikFaceBack{X}{X}{X}
	      {O}{O}{O}
	      {O}{O}{O}%
\ShowCube{2cm}{0.4}{\DrawRubikCubeRU}
\ShowCube{2.5cm}{0.4}{%
  \DrawRubikFaceUpSide
}\\[1em]
Eine Ecke ist richtig orientiert. Hier dreht man den Würfel so, dass diese Ecke vorne links liegt und führt dann den folgenden Algorithmus aus:
\begin{center}
	\sffamily\Large (\Algo{R, U, Rp, Up}) (\Algo{R, U, U, Rp})
\end{center}
Danach schaut man wieder welcher Fall nun vorliegt.

\pagebreak
\paragraph{Fall 3}
Beispiel:
\RubikCubeGreyAll%
\RubikFaceUp{Y}{Y}{Y}
            {Y}{Y}{Y}
            {X}{Y}{X}%
\RubikFaceFront{Y}{X}{X}
               {R}{R}{R}
	       {R}{R}{R}%
\RubikFaceRight{X}{X}{X}
	       {G}{G}{G}
	       {G}{G}{G}%
\RubikFaceLeft{X}{X}{X}
	      {B}{B}{B}
	      {B}{B}{B}%
\RubikFaceBack{X}{X}{X}
	      {O}{O}{O}
	      {O}{O}{O}%
\ShowCube{2cm}{0.4}{\DrawRubikCubeRU}
\ShowCube{2.5cm}{0.4}{%
  \DrawRubikFaceUpSide
}\\[1em]
Genau Zwei Ecken sind richtig orientiert. Dabei ist es unwichtig welches Muster sie konkret ergeben.
In diesem Fall dreht man den Würfel so, dass eine falsche Ecke vorne links liegt und mit
der gelben Seite zu einem zeigt. Dann wendet man wieder den Algorithmus aus Fall 2 an:
\begin{center}
	\sffamily\Large (\Algo{R, U, Rp, Up}) (\Algo{R, U, U, Rp})
\end{center}
Dann schaut man erneut welcher Fall nun vorliegt.

\paragraph{Fall 4}
Beispiel:
\RubikCubeGreyAll%
\RubikFaceUp{X}{Y}{X}
            {Y}{Y}{Y}
            {X}{Y}{X}%
\RubikFaceFront{X}{X}{X}
               {R}{R}{R}
	       {R}{R}{R}%
\RubikFaceRight{X}{X}{X}
	       {G}{G}{G}
	       {G}{G}{G}%
\RubikFaceLeft{X}{X}{Y}
	      {B}{B}{B}
	      {B}{B}{B}%
\RubikFaceBack{X}{X}{X}
	      {O}{O}{O}
	      {O}{O}{O}%
\ShowCube{2cm}{0.4}{\DrawRubikCubeRU}
\ShowCube{2.5cm}{0.4}{%
  \DrawRubikFaceUpSide
}\\[1em]
Keine Ecke ist richtig orientiert. Such Dir eine falsche Ecke und dreh den Würfel so,
dass sie vorne links liegt und mit ihrer gelben Seite nach links zeit. .
Wende wieder den Algorithmus aus Fall 2 an:
\begin{center}
	\sffamily\Large (\Algo{R, U, Rp, Up}) (\Algo{R, U, U, Rp})
\end{center}
Anschließend schaut man wieder welcher Fall vorliegt.

Sobald wir in Fall 1 angekommen sind haben wir das "'O"' von CFOP fertig und
können in die letzte Phase gehen.

\selectlanguage{ngerman}
\section{Positionierung der gelben Ecksteine}
Jetzt sind wir in der "'P"' Phase von CFOP angekommen, und müssen die gelben Steine so auf der obersten Schicht hin und her "'schieben"', dass alle Farben ringsherum korrekt sind.

Zunächst etwas Begriffsklärung: wenn zwei Ecksteine der letzten Ebene auf einer Seite die gleiche Farbe haben, dann nennt man diese Form einen "'Scheinwerfer"'.
Welche Farbe diese beiden Seiten dabei haben ist egal.

Im folgenden Beispiel sieht man zwei rote "'Scheinwerfer"' die sich auch zufällig auf der roten Seite befinden:
\begin{center}
  \RubikCubeGreyAll%
  \RubikFaceUp{Y}{Y}{Y}
	      {Y}{Y}{Y}
	      {Y}{Y}{Y}%
  \RubikFaceFront{R}{B}{R}
		 {R}{R}{R}
		 {R}{R}{R}%
  \RubikFaceRight{G}{R}{O}
		 {G}{G}{G}
		 {G}{G}{G}%
  \RubikFaceLeft{O}{O}{B}
		{B}{B}{B}
		{B}{B}{B}%
  \RubikFaceBack{G}{G}{B}
		{O}{O}{O}
		{O}{O}{O}%
  \ShowCube{2cm}{0.4}{\DrawRubikCubeRU}
  \ShowCube{2.5cm}{0.4}{\DrawRubikFaceUpSide}
\end{center}

Mit dieser Form lassen sich die folgenden Fälle beshreiben:

\paragraph{Fall 1}
\RubikCubeGreyAll%
\RubikFaceUp{Y}{Y}{Y}
            {Y}{Y}{Y}
            {Y}{Y}{Y}%
\RubikFaceFront{B}{X}{B}
               {R}{R}{R}
	       {R}{R}{R}%
\RubikFaceRight{R}{X}{R}
	       {G}{G}{G}
	       {G}{G}{G}%
\RubikFaceLeft{O}{X}{O}
	      {B}{B}{B}
	      {B}{B}{B}%
\RubikFaceBack{G}{X}{G}
	      {O}{O}{O}
	      {O}{O}{O}%
\ShowCube{2cm}{0.4}{\DrawRubikCubeRU}
\ShowCube{2.5cm}{0.4}{%
  \DrawRubikFaceUpSide
}\\[1em]
Hier sind Scheinwerfer sind auf allen vier Seiten.
Wir drehen jetzt die obere Schicht so lange bis die Scheinwerfer mit den Farben der Seite übereinstimmen und gehen dann zum nächsten Schritt.

\paragraph{Fall 2}
\RubikCubeGreyAll%
\RubikFaceUp{Y}{Y}{Y}
            {Y}{Y}{Y}
            {Y}{Y}{Y}%
\RubikFaceFront{X}{X}{X}
               {R}{R}{R}
	       {R}{R}{R}%
\RubikFaceRight{X}{X}{X}
	       {G}{G}{G}
	       {G}{G}{G}%
\RubikFaceLeft{R}{X}{R}
	      {B}{B}{B}
	      {B}{B}{B}%
\RubikFaceBack{X}{X}{X}
	      {O}{O}{O}
	      {O}{O}{O}%
\ShowCube{2cm}{0.4}{\DrawRubikCubeRU}
\ShowCube{2.5cm}{0.4}{%
  \DrawRubikFaceUpSide
}\\[1em]
Scheinwerfer sind auf \emph{einer einzigen} Seite. 
Hier drehen wir die Scheinwerfer zunächst so, dass sie nach links zeigen und wenden dann den folgenden Algorithmus an:
\begin{center}
	\sffamily\Large
	(\Algo{R, U, Rp, Up}) (\Algo{Rp, F, R}) \\[1em]
	(\Algo{R, Up, Rp, Up}) (\Algo{R, U, Rp, Fp})
\end{center}

\pagebreak
\paragraph{Fall 3}
\RubikCubeGreyAll%
\RubikFaceUp{Y}{Y}{Y}
            {Y}{Y}{Y}
            {Y}{Y}{Y}%
\RubikFaceFront{X}{X}{X}
               {R}{R}{R}
	       {R}{R}{R}%
\RubikFaceRight{X}{X}{X}
	       {G}{G}{G}
	       {G}{G}{G}%
\RubikFaceLeft{X}{X}{X}
	      {B}{B}{B}
	      {B}{B}{B}%
\RubikFaceBack{X}{X}{X}
	      {O}{O}{O}
	      {O}{O}{O}%
\ShowCube{2cm}{0.4}{\DrawRubikCubeRU}
\ShowCube{2.5cm}{0.4}{%
  \DrawRubikFaceUpSide
}\\[1em]
Es sind gar keine Scheinwerfer vorhanden. Wir wenden den selben Algorithmus
wie in Fall 2 an und schauen erneut:
\begin{center}
	\sffamily\Large
	(\Algo{R, U, Rp, Up}) (\Algo{Rp, F, R}) \\[1em]
	(\Algo{R, Up, Rp, Up}) (\Algo{R, U, Rp, Fp})
\end{center}

\section{Positionierung der gelben Kantensteine}
Dies ist die zweite Stufe des "'P"' Schrittes in CFOP.
Der Würfel müsste jetzt ungefähr so aussehen:
\begin{center}
  \RubikCubeGreyAll%
  \RubikFaceUp{Y}{Y}{Y}
	      {Y}{Y}{Y}
	      {Y}{Y}{Y}%
  \RubikFaceFront{R}{X}{R}
		 {R}{R}{R}
		 {R}{R}{R}%
  \RubikFaceRight{G}{X}{G}
		 {G}{G}{G}
		 {G}{G}{G}%
  \RubikFaceLeft{B}{X}{B}
		{B}{B}{B}
		{B}{B}{B}%
  \RubikFaceBack{O}{X}{O}
		{O}{O}{O}
		{O}{O}{O}%
  \ShowCube{2cm}{0.4}{\DrawRubikCubeRU}
  \ShowCube{2.5cm}{0.4}{\DrawRubikFaceUpSide}
\end{center}

Nur noch die gelben Kantensteine sind möglicherweise an der Falschen Position. Hier
haben wir wieder mehrere Fälle zu beachten:

\paragraph{Fall 1}
\RubikCubeGreyAll%
\RubikFaceUp{Y}{Y}{Y}
            {Y}{Y}{Y}
            {Y}{Y}{Y}%
\RubikFaceFront{R}{G}{R}
               {R}{R}{R}
	       {R}{R}{R}%
\RubikFaceRight{G}{B}{G}
	       {G}{G}{G}
	       {G}{G}{G}%
\RubikFaceLeft{B}{R}{B}
	      {B}{B}{B}
	      {B}{B}{B}%
\RubikFaceBack{O}{O}{O}
	      {O}{O}{O}
	      {O}{O}{O}%
\ShowCube{2cm}{0.4}{\DrawRubikCubeRU}
\ShowCube{2.5cm}{0.4}{%
  \DrawRubikFaceUpSide
}\\[1em]
Ein \emph{einziger} Kantenstein ist richtig. Wir halten jetzt den Würfel so, dass
die richtige Kante hinten liegt und wenden dann den folgenden Algorithmus an:
\begin{center}
	\sffamily\Large
	(\Algo{R, Up, R, U}) (\Algo{R, U, R, Up}) \\[1em] (\Algo{Rp, Up, R, R})
\end{center}

\paragraph{Fall 2}
\RubikCubeGreyAll%
\RubikFaceUp{Y}{Y}{Y}
            {Y}{Y}{Y}
            {Y}{Y}{Y}%
\RubikFaceFront{R}{G}{R}
               {R}{R}{R}
	       {R}{R}{R}%
\RubikFaceRight{G}{O}{G}
	       {G}{G}{G}
	       {G}{G}{G}%
\RubikFaceLeft{B}{R}{B}
	      {B}{B}{B}
	      {B}{B}{B}%
\RubikFaceBack{O}{B}{O}
	      {O}{O}{O}
	      {O}{O}{O}%
\ShowCube{2cm}{0.4}{\DrawRubikCubeRU}
\ShowCube{2.5cm}{0.4}{%
  \DrawRubikFaceUpSide
}\\[1em]
\emph{Kein} Kantenstein ist richtig. Hier wenden wir wieder den Algorithmus aus
Fall1 an und schauen dann erneut.
\begin{center}
	\sffamily\Large
	(\Algo{R, Up, R, U}) (\Algo{R, U, R, Up}) \\[1em] (\Algo{Rp, Up, R, R})
\end{center}

\paragraph{Fall 3}
\RubikCubeGreyAll%
\RubikFaceUp{Y}{Y}{Y}
            {Y}{Y}{Y}
            {Y}{Y}{Y}%
\RubikFaceFront{R}{R}{R}
               {R}{R}{R}
	       {R}{R}{R}%
\RubikFaceRight{G}{G}{G}
	       {G}{G}{G}
	       {G}{G}{G}%
\RubikFaceLeft{B}{B}{B}
	      {B}{B}{B}
	      {B}{B}{B}%
\RubikFaceBack{O}{O}{O}
	      {O}{O}{O}
	      {O}{O}{O}%
\ShowCube{2cm}{0.4}{\DrawRubikCubeRU}
\ShowCube{2.5cm}{0.4}{%
  \DrawRubikFaceUpSide
}\\[1em]
Alle vier Kanten sind richtig. Herzlichen Glückwunsch, der Würfel ist gelöst.

\section{Wie jetzt weiter?}
Möglicherweise liest Du das hier nachdem Du zum allerersten Mal den Würfel
mit Hilfe der Anleitung gelöst hast. Wenn Dich das Thema weiter interessiert,
dann empfehlen wir Dir, die Schritte bis jetzt einmal auswendig zu lernen.

\section{Tips und Tricks zum Auswendig lernen}

\begin{itemize}
	\item Lerne die Algortihmen der Reihenfolge der Schritte nach.
	\item Konzentrier Dich auf eine Algorithmus und mache erst weiter
	      wenn Du ihn kannst.
	\item Wenn Du einen Schritt fertig hast, kannst Du oft den selben
	      Algortihmus verwenden um den Schritt wieder rückgängig zu
	      machen. Das hilft Dir Dich auf den Algorithmus zu konzentrieren.
\end{itemize}

Jetzt wo Du diese Schritte und Algorithmen auswenig kannst, gehörst Du zu den
3\% der Weltbevölkerung die Rubik's Würfel ohne Hilfe lösen können!
Herzlichen Glückwunsch und willkommen im Club!

%
%\chapter{Fortgeschrittene Techniken}
%Die folgenden Techniken und Algorithmen sind für alle jenen, die eine Rubik's Würfel auswendig nach der
%oben beschriebenen Methode lösen können. Wir werden im folgenden weitere Algorithmen vorstellen,
%die Abkürzungen für Schritte aus der Anfängermethode darstellen.
%
%Die Anfängermethode ist dabei ein wichtiger Ausgangspunkt für alles folgende,
%und man sollte sie sicher anwenden können. Als Daumenregel sollte man ca. 20
%bis 50 Würfel auswendig (und ohne zu spicken) gelöst haben bevor man hier weiter macht.
%
%Du bist noch nicht so weit? Kein Problem! Leg los und löse ein paar Würfel! Wenn
%Du etwas Ansporn brauchst, dann stoppe dabei die Zeit und versuche schneller zu werden.
%Jetzt ist auch ein guter Zeitpunkt, um Dir ein paar Videos bezüglich Handhaltung und
%Fingertricks anzuschauen. Damit machen die fortgeschrittenen Techniken auch gleich mehr Spass.
%
%Wir warten hier auf Dich! Versprochen!
%
%\section{Noch mehr Rotationen}
Willkommen zurück. Für die nächsten Techniken brauchen wir ein paar Rotationen 
die es in der Anfängermethode so noch nicht gab. \\[1em]

\begin{instruction}{Rotating the vertical middle between L and R clockwise}
  \vspace{1.3em}
  \RubikCubeSolved%
  \ShowCube{15mm}{0.35}{\DrawRubikCube}%
  \quad\Rubik{Lm}\RubikRotation{Lm}
  \ShowCube{15mm}{0.35}{\DrawRubikCube}%
\end{instruction}
\hfil
\begin{instruction}{Rotating the vertical middle between L and R anticlockwise}
  \vspace{1.3em}
  \RubikCubeSolved%
  \ShowCube{15mm}{0.35}{\DrawRubikCube}%
  \quad\Rubik{Lmp}\RubikRotation{Lmp}
  \ShowCube{15mm}{0.35}{\DrawRubikCube}%
\end{instruction}\\[5em]
\begin{instruction}{Rotating the wide right side clockwise}
  \vspace{1.3em}
  \RubikCubeSolved%
  \ShowCube{15mm}{0.35}{\DrawRubikCube}%
  \quad\Rubik{Rw}\RubikRotation{Rw}
  \ShowCube{15mm}{0.35}{\DrawRubikCube}%
\end{instruction}
\hfil
\begin{instruction}{Rotating the wide right side anticlockwise}
  \vspace{1.3em}
  \RubikCubeSolved%
  \ShowCube{15mm}{0.35}{\DrawRubikCube}%
  \quad\Rubik{Rwp}\RubikRotation{Rwp}
  \ShowCube{15mm}{0.35}{\DrawRubikCube}%
\end{instruction}\\[5em]
\begin{instruction}{Rotating the wide front side clockwise}
  \vspace{1.3em}
  \RubikCubeSolved%
  \ShowCube{15mm}{0.35}{\DrawRubikCube}%
  \quad\Rubik{Fw}\RubikRotation{Fw}
  \ShowCube{15mm}{0.35}{\DrawRubikCube}%
\end{instruction}
\hfil
\begin{instruction}{Rotating the wide front side anticlockwise}
  \vspace{1.3em}
  \RubikCubeSolved%
  \ShowCube{15mm}{0.35}{\DrawRubikCube}%
  \quad\Rubik{Fwp}\RubikRotation{Fwp}
  \ShowCube{15mm}{0.35}{\DrawRubikCube}%
\end{instruction}\\[5em]

%\section{Eine Abkürzung für das gelbe Kreuz}

%\section{Fisch und Antifisch}

%
%% TODO: Jetzt vielleicht erst einmal Ua und Ub PLL
%% TODO: Dann vielleicht H-Perm und Z-Perm?????
%
%\section{Der T, U und L Fall}

%\section{Der Pi und H Fall}


\end{document}
