\selectlanguage{ngerman}
\section{Die gelbe Ebene in einem Schritt fertig ausrichten}
Beim gelben Kreuz haben wir fürs Erste alles Optimiert was geht.
Aber jetzt, auf dem Weg zur gelben Ebene, können wir einiges Optimieren.

Hat man jetzt das gelbe Kreuz fertig, können sich acht mögliche Muster ergeben.

Zwei für den Fall, dass keine gelbe Ecke korrekt ausgerichtet ist\footnote{Ist manchmal zufällig der Fall!}:
\parbox{0.5\linewidth}{
	\begin{center}
		\textbf{H}\\[1ex]
		\RubikCubeGreyAll%
		\RubikFaceUp{X}{Y}{X}
		{Y}{Y}{Y}
		{X}{Y}{X}%
		\RubikFaceFront{X}{X}{X}
			 {X}{X}{X}
			 {X}{X}{X}%
		\RubikFaceRight{Y}{X}{Y}
			 {X}{X}{X}
			 {X}{X}{X}%
		\RubikFaceLeft{Y}{X}{Y}
			{X}{X}{X}
			{X}{X}{X}%
		\RubikFaceBack{X}{X}{X}
			{X}{X}{X}
			{X}{X}{X}%
		%\ShowCube{2cm}{0.35}{\DrawRubikCubeRU}
		\ShowCube{1.5cm}{0.35}{%
			\DrawRubikFaceUpSide
		}
	\end{center}
}\parbox{0.5\linewidth}{
	\begin{center}
		\textbf{Pi}\\[1ex]
		\RubikCubeGreyAll%
		\RubikFaceUp{X}{Y}{X}
		{Y}{Y}{Y}
		{X}{Y}{X}%
		\RubikFaceFront{X}{X}{Y}
			 {X}{X}{X}
			 {X}{X}{X}%
		\RubikFaceRight{X}{X}{X}
			 {X}{X}{X}
			 {X}{X}{X}%
		\RubikFaceLeft{Y}{X}{Y}
			{X}{X}{X}
			{X}{X}{X}%
		\RubikFaceBack{Y}{X}{X}
			{X}{X}{X}
			{X}{X}{X}%
		%\ShowCube{2cm}{0.35}{\DrawRubikCubeRU}
		\ShowCube{1.5cm}{0.35}{%
			\DrawRubikFaceUpSide
		}
	\end{center}
}

Zwei für den Fall, dass eine gelbe Ecke korrekt ausgerichtet ist:
\parbox{0.5\linewidth}{
	\begin{center}
		\textbf{Anti-Fisch} \\[1ex]
		\RubikCubeGreyAll%
		\RubikFaceUp{X}{Y}{Y}
					{Y}{Y}{Y}
					{X}{Y}{X}%
		\RubikFaceFront{Y}{X}{X}
			 {X}{X}{X}
			 {X}{X}{X}%
		\RubikFaceRight{Y}{X}{X}
			 {X}{X}{X}
			 {X}{X}{X}%
		\RubikFaceLeft{Y}{X}{X}
			{X}{X}{X}
			{X}{X}{X}%
		\RubikFaceBack{X}{X}{X}
			{X}{X}{X}
			{X}{X}{X}%
		%\ShowCube{2cm}{0.35}{\DrawRubikCubeRU}
		\ShowCube{1.5cm}{0.35}{%
			\DrawRubikFaceUpSide
		}
	\end{center}
}\parbox{0.5\linewidth}{
	\begin{center}
		\textbf{Fisch} \\[1ex]
		\RubikCubeGreyAll%
		\RubikFaceUp{X}{Y}{X}
					{Y}{Y}{Y}
					{Y}{Y}{X}%
		\RubikFaceFront{X}{X}{Y}
			 {X}{X}{X}
			 {X}{X}{X}%
		\RubikFaceRight{X}{X}{Y}
			 {X}{X}{X}
			 {X}{X}{X}%
		\RubikFaceLeft{X}{X}{X}
			{X}{X}{X}
			{X}{X}{X}%
		\RubikFaceBack{X}{X}{Y}
			{X}{X}{X}
			{X}{X}{X}%
		%\ShowCube{2cm}{0.35}{\DrawRubikCubeRU}
		\ShowCube{1.5cm}{0.35}{%
			\DrawRubikFaceUpSide
		}
	\end{center}
}

Drei für den Fall, dass zwei gelbe Ecken korrekt ausgerichtet sind:
\parbox{0.3\linewidth}{
	\begin{center}
		\textbf{T} \\[1ex]
		\RubikCubeGreyAll%
		\RubikFaceUp{X}{Y}{Y}
					{Y}{Y}{Y}
					{X}{Y}{Y}%
		\RubikFaceFront{Y}{X}{X}
			 {X}{X}{X}
			 {X}{X}{X}%
		\RubikFaceRight{X}{X}{X}
			 {X}{X}{X}
			 {X}{X}{X}%
		\RubikFaceLeft{X}{X}{X}
			{X}{X}{X}
			{X}{X}{X}%
		\RubikFaceBack{X}{X}{Y}
			{X}{X}{X}
			{X}{X}{X}%
		%\ShowCube{2cm}{0.35}{\DrawRubikCubeRU}
		\ShowCube{1.5cm}{0.35}{%
			\DrawRubikFaceUpSide
		}
	\end{center}
}\parbox{0.3\linewidth}{
	\begin{center}
		\textbf{U} \\[1ex]
		\RubikCubeGreyAll%
		\RubikFaceUp{Y}{Y}{Y}
					{Y}{Y}{Y}
					{X}{Y}{X}%
		\RubikFaceFront{Y}{X}{Y}
			 {X}{X}{X}
			 {X}{X}{X}%
		\RubikFaceRight{X}{X}{X}
			 {X}{X}{X}
			 {X}{X}{X}%
		\RubikFaceLeft{X}{X}{X}
			{X}{X}{X}
			{X}{X}{X}%
		\RubikFaceBack{X}{X}{X}
			{X}{X}{X}
			{X}{X}{X}%
		%\ShowCube{2cm}{0.35}{\DrawRubikCubeRU}
		\ShowCube{1.5cm}{0.35}{%
			\DrawRubikFaceUpSide
		}
	\end{center}
}\parbox{0.3\linewidth}{
	\begin{center}
		\textbf{L} \\[1ex]
		\RubikCubeGreyAll%
		\RubikFaceUp{Y}{Y}{X}
					{Y}{Y}{Y}
					{X}{Y}{Y}%
		\RubikFaceFront{Y}{X}{X}
			 {X}{X}{X}
			 {X}{X}{X}%
		\RubikFaceRight{X}{X}{Y}
			 {X}{X}{X}
			 {X}{X}{X}%
		\RubikFaceLeft{X}{X}{X}
			{X}{X}{X}
			{X}{X}{X}%
		\RubikFaceBack{X}{X}{X}
			{X}{X}{X}
			{X}{X}{X}%
		%\ShowCube{2cm}{0.35}{\DrawRubikCubeRU}
		\ShowCube{1.5cm}{0.35}{%
			\DrawRubikFaceUpSide
		}
	\end{center}
}

Eins für den Fall, dass alle gelben Ecken korrekt ausgerichtet sind:
\parbox{\linewidth}{
\begin{center}
  \textbf{Fertig} \\[1ex]
  \RubikCubeGreyAll%
  \RubikFaceUp{Y}{Y}{Y}
	      {Y}{Y}{Y}
	      {Y}{Y}{Y}%
  \RubikFaceFront{X}{X}{X}
		 {X}{X}{X}
		 {X}{X}{X}%
  \RubikFaceRight{X}{X}{X}
		 {X}{X}{X}
		 {X}{X}{X}%
  \RubikFaceLeft{X}{X}{X}
		{X}{X}{X}
		{X}{X}{X}%
  \RubikFaceBack{X}{X}{X}
		{X}{X}{X}
		{X}{X}{X}%
  %\ShowCube{2cm}{0.35}{\DrawRubikCubeRU}
  \ShowCube{1.5cm}{0.35}{%
    \DrawRubikFaceUpSide
  }
\end{center}
}\\[3em]

In der Anfängermethode sind uns die Muster bereits begegnet, wir haben sie nur nicht bewusst wahrgenommen.
Und wir haben in der Anfängermethode bereits den Fall gelernt, um aus dem \textbf{Fisch}-Muster die fertige gelbe Fläche zu machen.
Zusätzlich nutzen wir aber auch diesen Algorithmus, um die anderen Muster Schritt für Schritt in das \textbf{Fisch}-Muster zu überführen:
\begin{center}
	\begin{tikzpicture}[nodes={draw, thick, rounded corners, font=\bfseries}]
		\node[] (T) {T};
		\node[] (U) [above=of T] {U};
		\node[] (L) [below=of T] {L};
		\node[] (Antifisch) [right=of T] {Anti-Fisch};
		\node[] (Fisch) [right=of Antifisch] {Fisch};
		\node[] (Pi) [above=of Fisch] {Pi};
		\node[] (H) [below=of Fisch] {H};
		\node[] (Fertig) [right=of Fisch] {Fertig};

		\draw[->] (T) to [out=0, in=180] (Antifisch);
		\draw[->] (U) to [out=0, in=90] (Antifisch);
		\draw[->] (L) to [out=0, in=270] (Antifisch);
		\draw[->] (Antifisch) to (Fisch);
		\draw[->] (Pi) to [out=270, in=90] (Fisch);
		\draw[->] (H) to (Fisch);
		\draw[->] (Fisch) to (Fertig);
	\end{tikzpicture}
\end{center}

Hier können wir jetzt immer mehr Abkürzungen einbauen.
Denn es gibt für jedes der sechs übrigen Muster einen Algorithmus, mit denen man aus ihnen die gelbe Fläche machen kann.
Diese kannst Du jetzt sukzessive lernen.
Hast Du für ein Muster den Algorithmus noch nicht auswendig gelernt, greif einfach auf die Anfängermethode zurück, die ist voll kompatibel!
Mit welchem Muster Du anfangen willst, das ist Dir überlassen.

Damit Du einen Algorithmus auch effektiv üben kannst, musst Du immer wieder das passende Muster herstellen.
Dafür haben wir Dir zusätzlich immer noch einen sogenannten "'Training Scramble"' unter dem Algorithmus abgedruckt.
Nimm dazu einen gelösten Würfel und halte ihn mit der gelben Ebene nach oben.
Führe dann die Drehungen aus dem "'Training Scramble"' aus, um das gesuchte Muster zum Üben herzustellen.

Wenn Du bei einer gelben Fläche anfängst und einen Algorithmus zwei bis drei Mal wiederholst wirst Du auch wieder bei der gelben Fläche ankommen.
Damit kann man auch ohne einen "'Training Scramble"' einen Algorithmus wiederholen, bis man ihn sicher beherrscht.
Am besten ist es zudem, wenn so ein Algorithmus im Muskelgedächtnis verankert ist.
Dann muss man kaum bewusst darüber nachdenken, denn die Finger wissen quasi schon selber was sie zu tun haben.
\\[1em]
%
\parbox{0.3\linewidth}{
	\centering
	\textbf{Fisch} \\[1ex]
  \RubikCubeGreyAll%
  \RubikFaceUp{X}{Y}{X}
	      {Y}{Y}{Y}
	      {Y}{Y}{X}%
  \RubikFaceFront{X}{X}{Y}
		 {X}{X}{X}
		 {X}{X}{X}%
  \RubikFaceRight{X}{X}{Y}
		 {X}{X}{X}
		 {X}{X}{X}%
  \RubikFaceLeft{X}{X}{X}
		{X}{X}{X}
		{X}{X}{X}%
  \RubikFaceBack{X}{X}{Y}
		{X}{X}{X}
		{X}{X}{X}%
  %\ShowCube{2cm}{0.35}{\DrawRubikCubeRU}
  \ShowCube{\linewidth}{0.35}{%
    \DrawRubikFaceUpSide
  }
}%
\parbox{0.7\linewidth}{
	(\Algo{R, U, Rp, U}) \\[1em] (\Algo{R, U, U, Rp})
}\\[1em]
\Train{Up, Bp, Up, B, Up, Bp, U2, B, Up, B2, R2, L, U, L, Up L2, D, L2, Dp, R2, B2} \\[3em]
%
%
\parbox{0.3\linewidth}{
\begin{center}
	\textbf{Anti-Fisch} \\[1ex]
  \RubikCubeGreyAll%
  \RubikFaceUp{X}{Y}{Y}
	      {Y}{Y}{Y}
	      {X}{Y}{X}%
  \RubikFaceFront{Y}{X}{X}
		 {X}{X}{X}
		 {X}{X}{X}%
  \RubikFaceRight{Y}{X}{X}
		 {X}{X}{X}
		 {X}{X}{X}%
  \RubikFaceLeft{Y}{X}{X}
		{X}{X}{X}
		{X}{X}{X}%
  \RubikFaceBack{X}{X}{X}
		{X}{X}{X}
		{X}{X}{X}%
  %\ShowCube{2cm}{0.35}{\DrawRubikCubeRU}
  \ShowCube{1.5cm}{0.35}{%
    \DrawRubikFaceUpSide
  }
\end{center}
}%
\parbox{0.7\linewidth}{
	(\Algo{R, U, U, Rp, Up}) \\[1em] (\Algo{R, Up, Rp})
}\\[1em]
\Train{U' L' D' R D L' D' R U R2 U' L2 F2 U' L2 B2 L2 F2 R2}\\[3em]
%
%
\parbox{0.3\linewidth}{
\begin{center}
  \textbf{H} \\
  \RubikCubeGreyAll%
  \RubikFaceUp{X}{Y}{X}
	      {Y}{Y}{Y}
	      {X}{Y}{X}%
  \RubikFaceFront{X}{X}{X}
		 {X}{X}{X}
		 {X}{X}{X}%
  \RubikFaceRight{Y}{X}{Y}
		 {X}{X}{X}
		 {X}{X}{X}%
  \RubikFaceLeft{Y}{X}{Y}
		{X}{X}{X}
		{X}{X}{X}%
  \RubikFaceBack{X}{X}{X}
		{X}{X}{X}
		{X}{X}{X}%
  %\ShowCube{2cm}{0.35}{\DrawRubikCubeRU}
  \ShowCube{1.5cm}{0.35}{%
    \DrawRubikFaceUpSide
  }
\end{center}
}%
\parbox{0.7\linewidth}{
	(\Algo{R, U, Rp, U}) \\[1em]
	(\Algo{R, Up, Rp, U}) \\[1em]
  (\Algo{R, U, U, Rp})
}\\[1em]
\Train{U' L B L' B2 R B' R B2 L2 F2 U2 B2 D' B2 U L2 B2 D' B2}\\[3em]
%
%
\parbox{0.3\linewidth}{
\begin{center}
  \textbf{L} \\
  \RubikCubeGreyAll%
  \RubikFaceUp{Y}{Y}{X}
	      {Y}{Y}{Y}
	      {X}{Y}{Y}%
  \RubikFaceFront{Y}{X}{X}
		 {X}{X}{X}
		 {X}{X}{X}%
  \RubikFaceRight{X}{X}{Y}
		 {X}{X}{X}
		 {X}{X}{X}%
  \RubikFaceLeft{X}{X}{X}
		{X}{X}{X}
		{X}{X}{X}%
  \RubikFaceBack{X}{X}{X}
		{X}{X}{X}
		{X}{X}{X}%
  %\ShowCube{2cm}{0.35}{\DrawRubikCubeRU}
  \ShowCube{1.5cm}{0.35}{%
    \DrawRubikFaceUpSide
  }
\end{center}
}%
\parbox{0.7\linewidth}{
	(\Algo{F, Rp, Fp, Rw}) \\[1em] (\Algo{U, R, Up, Rwp})
}\\[1em]
\Train{U L U' F2 U F2 U L R2 U' L2 U' L2 U2 L2 F2 U F2 U R2}\\[3em]
%
%
\parbox{0.3\linewidth}{
\begin{center}
  \textbf{Pi} \\
  \RubikCubeGreyAll%
  \RubikFaceUp{X}{Y}{X}
	      {Y}{Y}{Y}
	      {X}{Y}{X}%
  \RubikFaceFront{X}{X}{Y}
		 {X}{X}{X}
		 {X}{X}{X}%
  \RubikFaceRight{X}{X}{X}
		 {X}{X}{X}
		 {X}{X}{X}%
  \RubikFaceLeft{Y}{X}{Y}
		{X}{X}{X}
		{X}{X}{X}%
  \RubikFaceBack{Y}{X}{X}
		{X}{X}{X}
		{X}{X}{X}%
  %\ShowCube{2cm}{0.35}{\DrawRubikCubeRU}
  \ShowCube{1.5cm}{0.35}{%
    \DrawRubikFaceUpSide
  }
\end{center}
}%
\parbox{0.7\linewidth}{
	\Algo{R, U, U} \\[1em]
	(\Algo{R, R, Up}) (\Algo{R, R, Up}) \\[1em]
  \Algo{R, R, U, U, R}
}\\[1em]
\Train{U2 F U2 F2 L2 F' L2 U2 F U2 L2 U' B2 U F2 U' B2 L2 U L2}\\[3em]
%
%
\parbox{0.3\linewidth}{
\begin{center}
  \textbf{T} \\
  \RubikCubeGreyAll%
  \RubikFaceUp{X}{Y}{Y}
	      {Y}{Y}{Y}
	      {X}{Y}{Y}%
  \RubikFaceFront{Y}{X}{X}
		 {X}{X}{X}
		 {X}{X}{X}%
  \RubikFaceRight{X}{X}{X}
		 {X}{X}{X}
		 {X}{X}{X}%
  \RubikFaceLeft{X}{X}{X}
		{X}{X}{X}
		{X}{X}{X}%
  \RubikFaceBack{X}{X}{Y}
		{X}{X}{X}
		{X}{X}{X}%
  %\ShowCube{2cm}{0.35}{\DrawRubikCubeRU}
  \ShowCube{1.5cm}{0.35}{%
    \DrawRubikFaceUpSide
  }
\end{center}
}%
\parbox{0.7\linewidth}{
	(\Algo{Rw, U, Rp, Up}) \\[1em] (\Algo{Rwp, F, R, Fp})
}\\[1em]
\Train{U F D B2 D' F U2 B2 U B2 U R2 U R2 U B2 F2}\\[3em]
%
%
\parbox{0.3\linewidth}{
\begin{center}
  \textbf{U} \\
  \RubikCubeGreyAll%
  \RubikFaceUp{Y}{Y}{Y}
	      {Y}{Y}{Y}
	      {X}{Y}{X}%
  \RubikFaceFront{Y}{X}{Y}
		 {X}{X}{X}
		 {X}{X}{X}%
  \RubikFaceRight{X}{X}{X}
		 {X}{X}{X}
		 {X}{X}{X}%
  \RubikFaceLeft{X}{X}{X}
		{X}{X}{X}
		{X}{X}{X}%
  \RubikFaceBack{X}{X}{X}
		{X}{X}{X}
		{X}{X}{X}%
  %\ShowCube{2cm}{0.35}{\DrawRubikCubeRU}
  \ShowCube{1.5cm}{0.35}{%
    \DrawRubikFaceUpSide
  }
\end{center}
}%
\parbox{0.7\linewidth}{
	\Algo{R, R, D} \\[1em] (\Algo{Rp, U, U, R}) \\[1em]
	\Algo{Dp} \\[1em]
	(\Algo{Rp, U, U, Rp})
}\\[1em]
\Train{U L U' F2 U F2 U L U2 F2 D R2 B2 U B2 D' R2 F2 U L2}
