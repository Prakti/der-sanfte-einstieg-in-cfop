\selectlanguage{ngerman}
\section{Die gelbe Ebene in einem Schritt fertig ausrichten}
Nachdem das gelbe Kreuz fertig ist, ergeben sich sieben mögliche Muster, die alle miteinander zusammenhängen.

In der Anfängermethode haben wir gelernt wie wir diese Muster mit einem Algorithmus ineinander überführen, bis wir bei der fertigen Ebene 'raus kommen.
Konkret hängen diese Muster folgendermaßen miteinander zusammen:\\[1em]
\parbox{0.2\linewidth}{
  \centering
  \textbf{T} \\[1ex]
  \RubikCubeGreyAll%
  \RubikFaceUp{X}{Y}{Y}
	      {Y}{Y}{Y}
	      {X}{Y}{Y}%
  \RubikFaceFront{Y}{X}{X}
		 {X}{X}{X}
		 {X}{X}{X}%
  \RubikFaceRight{X}{X}{X}
		 {X}{X}{X}
		 {X}{X}{X}%
  \RubikFaceLeft{X}{X}{X}
		{X}{X}{X}
		{X}{X}{X}%
  \RubikFaceBack{X}{X}{Y}
		{X}{X}{X}
		{X}{X}{X}%
  %\ShowCube{2cm}{0.35}{\DrawRubikCubeRU}
  \ShowCube{1.5cm}{0.35}{%
    \DrawRubikFaceUpSide
  }\\[1em]
  \textbf{U} \\[1ex]
  \RubikCubeGreyAll%
  \RubikFaceUp{Y}{Y}{Y}
	      {Y}{Y}{Y}
	      {X}{Y}{X}%
  \RubikFaceFront{Y}{X}{Y}
		 {X}{X}{X}
		 {X}{X}{X}%
  \RubikFaceRight{X}{X}{X}
		 {X}{X}{X}
		 {X}{X}{X}%
  \RubikFaceLeft{X}{X}{X}
		{X}{X}{X}
		{X}{X}{X}%
  \RubikFaceBack{X}{X}{X}
		{X}{X}{X}
		{X}{X}{X}%
  %\ShowCube{2cm}{0.35}{\DrawRubikCubeRU}
  \ShowCube{1.5cm}{0.35}{%
    \DrawRubikFaceUpSide
  }\\[1em]
  \textbf{L} \\[1ex]
  \RubikCubeGreyAll%
  \RubikFaceUp{Y}{Y}{X}
	      {Y}{Y}{Y}
	      {X}{Y}{Y}%
  \RubikFaceFront{Y}{X}{X}
		 {X}{X}{X}
		 {X}{X}{X}%
  \RubikFaceRight{X}{X}{Y}
		 {X}{X}{X}
		 {X}{X}{X}%
  \RubikFaceLeft{X}{X}{X}
		{X}{X}{X}
		{X}{X}{X}%
  \RubikFaceBack{X}{X}{X}
		{X}{X}{X}
		{X}{X}{X}%
  %\ShowCube{2cm}{0.35}{\DrawRubikCubeRU}
  \ShowCube{1.5cm}{0.35}{%
    \DrawRubikFaceUpSide
  }
}
$\Longrightarrow$
\parbox{0.2\linewidth}{
\begin{center}
  \textbf{Anti-Fisch} \\[1ex]
  \RubikCubeGreyAll%
  \RubikFaceUp{X}{Y}{Y}
	      {Y}{Y}{Y}
	      {X}{Y}{X}%
  \RubikFaceFront{Y}{X}{X}
		 {X}{X}{X}
		 {X}{X}{X}%
  \RubikFaceRight{Y}{X}{X}
		 {X}{X}{X}
		 {X}{X}{X}%
  \RubikFaceLeft{Y}{X}{X}
		{X}{X}{X}
		{X}{X}{X}%
  \RubikFaceBack{X}{X}{X}
		{X}{X}{X}
		{X}{X}{X}%
  %\ShowCube{2cm}{0.35}{\DrawRubikCubeRU}
  \ShowCube{1.5cm}{0.35}{%
    \DrawRubikFaceUpSide
  }
\end{center}
}
$\Longrightarrow$
\parbox{0.2\linewidth}{
\begin{center}
  \textbf{Fisch} \\[1ex]
  \RubikCubeGreyAll%
  \RubikFaceUp{X}{Y}{X}
	      {Y}{Y}{Y}
	      {Y}{Y}{X}%
  \RubikFaceFront{X}{X}{Y}
		 {X}{X}{X}
		 {X}{X}{X}%
  \RubikFaceRight{X}{X}{Y}
		 {X}{X}{X}
		 {X}{X}{X}%
  \RubikFaceLeft{X}{X}{X}
		{X}{X}{X}
		{X}{X}{X}%
  \RubikFaceBack{X}{X}{Y}
		{X}{X}{X}
		{X}{X}{X}%
  %\ShowCube{2cm}{0.35}{\DrawRubikCubeRU}
  \ShowCube{1.5cm}{0.35}{%
    \DrawRubikFaceUpSide
  }
\end{center}
}
$\Longrightarrow$
\parbox{0.2\linewidth}{
\begin{center}
  \textbf{Fertig} \\[1ex]
  \RubikCubeGreyAll%
  \RubikFaceUp{Y}{Y}{Y}
	      {Y}{Y}{Y}
	      {Y}{Y}{Y}%
  \RubikFaceFront{X}{X}{X}
		 {X}{X}{X}
		 {X}{X}{X}%
  \RubikFaceRight{X}{X}{Y}
		 {X}{X}{X}
		 {X}{X}{X}%
  \RubikFaceLeft{X}{X}{X}
		{X}{X}{X}
		{X}{X}{X}%
  \RubikFaceBack{X}{X}{X}
		{X}{X}{X}
		{X}{X}{X}%
  %\ShowCube{2cm}{0.35}{\DrawRubikCubeRU}
  \ShowCube{1.5cm}{0.35}{%
    \DrawRubikFaceUpSide
  }
\end{center}
}\\[3em]
Und dann auch noch so:\\[1em]
\parbox{0.2\linewidth}{
  \begin{center}
    \textbf{H} \\[1ex]
    \RubikCubeGreyAll%
    \RubikFaceUp{X}{Y}{X}
		{Y}{Y}{Y}
		{X}{Y}{X}%
    \RubikFaceFront{X}{X}{X}
		   {X}{X}{X}
		   {X}{X}{X}%
    \RubikFaceRight{Y}{X}{Y}
		   {X}{X}{X}
		   {X}{X}{X}%
    \RubikFaceLeft{Y}{X}{Y}
		  {X}{X}{X}
		  {X}{X}{X}%
    \RubikFaceBack{X}{X}{X}
		  {X}{X}{X}
		  {X}{X}{X}%
    %\ShowCube{2cm}{0.35}{\DrawRubikCubeRU}
    \ShowCube{1.5cm}{0.35}{%
      \DrawRubikFaceUpSide
    }
  \end{center}
  \begin{center}
    \textbf{Pi} \\[1ex]
    \RubikCubeGreyAll%
    \RubikFaceUp{X}{Y}{X}
		{Y}{Y}{Y}
		{X}{Y}{X}%
    \RubikFaceFront{X}{X}{Y}
		   {X}{X}{X}
		   {X}{X}{X}%
    \RubikFaceRight{X}{X}{X}
		   {X}{X}{X}
		   {X}{X}{X}%
    \RubikFaceLeft{Y}{X}{Y}
		  {X}{X}{X}
		  {X}{X}{X}%
    \RubikFaceBack{Y}{X}{X}
		  {X}{X}{X}
		  {X}{X}{X}%
    %\ShowCube{2cm}{0.35}{\DrawRubikCubeRU}
    \ShowCube{1.5cm}{0.35}{%
      \DrawRubikFaceUpSide
    }
  \end{center}
}
$\Longrightarrow$
\parbox{0.2\linewidth}{
\begin{center}
  \textbf{Fisch} \\[1ex]
  \RubikCubeGreyAll%
  \RubikFaceUp{X}{Y}{X}
	      {Y}{Y}{Y}
	      {Y}{Y}{X}%
  \RubikFaceFront{X}{X}{Y}
		 {X}{X}{X}
		 {X}{X}{X}%
  \RubikFaceRight{X}{X}{Y}
		 {X}{X}{X}
		 {X}{X}{X}%
  \RubikFaceLeft{X}{X}{X}
		{X}{X}{X}
		{X}{X}{X}%
  \RubikFaceBack{X}{X}{Y}
		{X}{X}{X}
		{X}{X}{X}%
  %\ShowCube{2cm}{0.35}{\DrawRubikCubeRU}
  \ShowCube{1.5cm}{0.35}{%
    \DrawRubikFaceUpSide
  }
\end{center}
}
$\Longrightarrow$
\parbox{0.2\linewidth}{
\begin{center}
  \textbf{Fertig} \\[1ex]
  \RubikCubeGreyAll%
  \RubikFaceUp{Y}{Y}{Y}
	      {Y}{Y}{Y}
	      {Y}{Y}{Y}%
  \RubikFaceFront{X}{X}{X}
		 {X}{X}{X}
		 {X}{X}{X}%
  \RubikFaceRight{X}{X}{Y}
		 {X}{X}{X}
		 {X}{X}{X}%
  \RubikFaceLeft{X}{X}{X}
		{X}{X}{X}
		{X}{X}{X}%
  \RubikFaceBack{X}{X}{X}
		{X}{X}{X}
		{X}{X}{X}%
  %\ShowCube{2cm}{0.35}{\DrawRubikCubeRU}
  \ShowCube{1.5cm}{0.35}{%
    \DrawRubikFaceUpSide
  }
\end{center}
}\\[2em]

Den Algorithmus, um vom \textbf{Fisch}-Muster zur fertigen Fläche zu kommen, haben wir bereits in der Anfängermethode gelernt.
Wir können jetzt Stück für Stück weitere Algorithmen dazu nehmen, welche dann auch aus einem der Muster direkt die fertige Fläche erzeugen.

Die Muster \textbf{T}, \textbf{U} und \textbf{L} sind nach der Anfängermethode am weitesten von der fertigen Fläche weg.
Hier kann man also die meisten Schritte sparen, wenn man die dazu gehörigen Algorithmen lernt.
Anschließend lohnt es sich die \textbf{H} und \textbf{Pi} Fälle zu lernen und zum Schluss den \textbf{Anti-Fisch}.
