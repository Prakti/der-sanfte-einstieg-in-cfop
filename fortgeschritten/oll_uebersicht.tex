\selectlanguage{ngerman}
\section{Die gelbe Ebene in einem Schritt fertig ausrichten}
Auf dem Weg vom gelben Kreuz zur gelben Ebene können wir schneller werden, indem wir durch die Anwendung von nur einem Algorithmus zur gelben Ebene kommen.
Hat man jetzt das gelbe Kreuz fertig, können sich (nur) acht mögliche Muster ergeben:

Zwei für den Fall, dass keine gelbe Ecke korrekt ausgerichtet ist:
\parbox{0.5\linewidth}{
	\begin{center}
		\textbf{H}\\[1ex]
		\RubikCubeGreyAll%
		\RubikFaceUp{X}{Y}{X}
		{Y}{Y}{Y}
		{X}{Y}{X}%
		\RubikFaceFront{X}{X}{X}
			 {X}{X}{X}
			 {X}{X}{X}%
		\RubikFaceRight{Y}{X}{Y}
			 {X}{X}{X}
			 {X}{X}{X}%
		\RubikFaceLeft{Y}{X}{Y}
			{X}{X}{X}
			{X}{X}{X}%
		\RubikFaceBack{X}{X}{X}
			{X}{X}{X}
			{X}{X}{X}%
		%\ShowCube{2cm}{0.35}{\DrawRubikCubeRU}
		\ShowCube{1.5cm}{0.35}{%
			\DrawRubikFaceUpSide
		}
	\end{center}
}\parbox{0.5\linewidth}{
	\begin{center}
		\textbf{Pi}\\[1ex]
		\RubikCubeGreyAll%
		\RubikFaceUp{X}{Y}{X}
		{Y}{Y}{Y}
		{X}{Y}{X}%
		\RubikFaceFront{X}{X}{Y}
			 {X}{X}{X}
			 {X}{X}{X}%
		\RubikFaceRight{X}{X}{X}
			 {X}{X}{X}
			 {X}{X}{X}%
		\RubikFaceLeft{Y}{X}{Y}
			{X}{X}{X}
			{X}{X}{X}%
		\RubikFaceBack{Y}{X}{X}
			{X}{X}{X}
			{X}{X}{X}%
		%\ShowCube{2cm}{0.35}{\DrawRubikCubeRU}
		\ShowCube{1.5cm}{0.35}{%
			\DrawRubikFaceUpSide
		}
	\end{center}
}

Zwei für den Fall, dass eine gelbe Ecke korrekt ausgerichtet ist:
\parbox{0.5\linewidth}{
	\begin{center}
		\textbf{Anti-Fisch} \\[1ex]
		\RubikCubeGreyAll%
		\RubikFaceUp{X}{Y}{Y}
					{Y}{Y}{Y}
					{X}{Y}{X}%
		\RubikFaceFront{Y}{X}{X}
			 {X}{X}{X}
			 {X}{X}{X}%
		\RubikFaceRight{Y}{X}{X}
			 {X}{X}{X}
			 {X}{X}{X}%
		\RubikFaceLeft{Y}{X}{X}
			{X}{X}{X}
			{X}{X}{X}%
		\RubikFaceBack{X}{X}{X}
			{X}{X}{X}
			{X}{X}{X}%
		%\ShowCube{2cm}{0.35}{\DrawRubikCubeRU}
		\ShowCube{1.5cm}{0.35}{%
			\DrawRubikFaceUpSide
		}
	\end{center}
}\parbox{0.5\linewidth}{
	\begin{center}
		\textbf{Fisch} \\[1ex]
		\RubikCubeGreyAll%
		\RubikFaceUp{X}{Y}{X}
					{Y}{Y}{Y}
					{Y}{Y}{X}%
		\RubikFaceFront{X}{X}{Y}
			 {X}{X}{X}
			 {X}{X}{X}%
		\RubikFaceRight{X}{X}{Y}
			 {X}{X}{X}
			 {X}{X}{X}%
		\RubikFaceLeft{X}{X}{X}
			{X}{X}{X}
			{X}{X}{X}%
		\RubikFaceBack{X}{X}{Y}
			{X}{X}{X}
			{X}{X}{X}%
		%\ShowCube{2cm}{0.35}{\DrawRubikCubeRU}
		\ShowCube{1.5cm}{0.35}{%
			\DrawRubikFaceUpSide
		}
	\end{center}
}

Drei für den Fall, dass zwei gelbe Ecken korrekt ausgerichtet sind:
\parbox{0.3\linewidth}{
	\begin{center}
		\textbf{T} \\[1ex]
		\RubikCubeGreyAll%
		\RubikFaceUp{X}{Y}{Y}
					{Y}{Y}{Y}
					{X}{Y}{Y}%
		\RubikFaceFront{Y}{X}{X}
			 {X}{X}{X}
			 {X}{X}{X}%
		\RubikFaceRight{X}{X}{X}
			 {X}{X}{X}
			 {X}{X}{X}%
		\RubikFaceLeft{X}{X}{X}
			{X}{X}{X}
			{X}{X}{X}%
		\RubikFaceBack{X}{X}{Y}
			{X}{X}{X}
			{X}{X}{X}%
		%\ShowCube{2cm}{0.35}{\DrawRubikCubeRU}
		\ShowCube{1.5cm}{0.35}{%
			\DrawRubikFaceUpSide
		}
	\end{center}
}\parbox{0.3\linewidth}{
	\begin{center}
		\textbf{U} \\[1ex]
		\RubikCubeGreyAll%
		\RubikFaceUp{Y}{Y}{Y}
					{Y}{Y}{Y}
					{X}{Y}{X}%
		\RubikFaceFront{Y}{X}{Y}
			 {X}{X}{X}
			 {X}{X}{X}%
		\RubikFaceRight{X}{X}{X}
			 {X}{X}{X}
			 {X}{X}{X}%
		\RubikFaceLeft{X}{X}{X}
			{X}{X}{X}
			{X}{X}{X}%
		\RubikFaceBack{X}{X}{X}
			{X}{X}{X}
			{X}{X}{X}%
		%\ShowCube{2cm}{0.35}{\DrawRubikCubeRU}
		\ShowCube{1.5cm}{0.35}{%
			\DrawRubikFaceUpSide
		}
	\end{center}
}\parbox{0.3\linewidth}{
	\begin{center}
		\textbf{L} \\[1ex]
		\RubikCubeGreyAll%
		\RubikFaceUp{Y}{Y}{X}
					{Y}{Y}{Y}
					{X}{Y}{Y}%
		\RubikFaceFront{Y}{X}{X}
			 {X}{X}{X}
			 {X}{X}{X}%
		\RubikFaceRight{X}{X}{Y}
			 {X}{X}{X}
			 {X}{X}{X}%
		\RubikFaceLeft{X}{X}{X}
			{X}{X}{X}
			{X}{X}{X}%
		\RubikFaceBack{X}{X}{X}
			{X}{X}{X}
			{X}{X}{X}%
		%\ShowCube{2cm}{0.35}{\DrawRubikCubeRU}
		\ShowCube{1.5cm}{0.35}{%
			\DrawRubikFaceUpSide
		}
	\end{center}
}

Eins für den Fall, dass alle gelben Ecken korrekt ausgerichtet sind:
\parbox{\linewidth}{
\begin{center}
  \textbf{Fertig} \\[1ex]
  \RubikCubeGreyAll%
  \RubikFaceUp{Y}{Y}{Y}
	      {Y}{Y}{Y}
	      {Y}{Y}{Y}%
  \RubikFaceFront{X}{X}{X}
		 {X}{X}{X}
		 {X}{X}{X}%
  \RubikFaceRight{X}{X}{X}
		 {X}{X}{X}
		 {X}{X}{X}%
  \RubikFaceLeft{X}{X}{X}
		{X}{X}{X}
		{X}{X}{X}%
  \RubikFaceBack{X}{X}{X}
		{X}{X}{X}
		{X}{X}{X}%
  %\ShowCube{2cm}{0.35}{\DrawRubikCubeRU}
  \ShowCube{1.5cm}{0.35}{%
    \DrawRubikFaceUpSide
  }
\end{center}
}\\[3em]

In der Anfängermethode sind uns die Muster bereits begegnet, wir haben sie nur nicht bewusst wahrgenommen.
Hier einmal ein Beispiel wie man mit der Anfängermethode durch die obigen Muster geht um zu der gelben Fläche zu kommen:

\newcommand{\OLLU}{
	\RubikCubeGreyAll%
	\RubikFaceUp{Y}{Y}{Y}
				{Y}{Y}{Y}
				{X}{Y}{X}%
	\RubikFaceFront{Y}{X}{Y}
		 {X}{X}{X}
		 {X}{X}{X}%
	\RubikFaceRight{X}{X}{X}
		 {X}{X}{X}
		 {X}{X}{X}%
	\RubikFaceLeft{X}{X}{X}
		{X}{X}{X}
		{X}{X}{X}%
	\RubikFaceBack{X}{X}{X}
		{X}{X}{X}
		{X}{X}{X}%
}

\parbox{0.7\linewidth}{
	Angenommen wir erhalten das \textbf{U} Muster, nachdem wir das gelbe Kreuz gebildet haben:
}\parbox{0.3\linewidth}{
	\centering
	\OLLU
	\ShowCube{1.5cm}{0.35}{%
		\DrawRubikFaceUpSide
	}
}\\[1em]
Das entspricht Fall 3 aus der Anfängermethode, wir müssen den Würfel nicht drehen, weil bereits eine falsche Ecke vorne links liegt und können den Algorithmus direkt anwenden:\\[1em]
(\Algo{R, U, Rp, U}) (\Algo{R, U, U, Rp})\\[1em]
\parbox{0.7\linewidth}{
	Dann erhalten wir das \textbf{Anti-Fisch} Muster, allerdings um 90° gedreht. Das entspricht, Fall 2 aus der Anfängermethode. }
\parbox{0.3\linewidth}{
	\centering
	\OLLU
	\RubikRotation{R, U, Rp, U, R, U, U, Rp}
	\ShowCube{15mm}{0.35}{\DrawRubikFaceUpSide}\\
}\\[1em]
\parbox{0.7\linewidth}{
	Wir müssen also erst den Würfel so drehen, dass die korrekte Ecke links unten ist:
}\parbox{0.3\linewidth}{
	\centering
	\centering
	\OLLU
	\RubikRotation{R, U, Rp, U, R, U, U, Rp}
	\RubikRotation{yp}
	\ShowCube{15mm}{0.35}{\DrawRubikFaceUpSide}\\
}\\[1em]
Jetzt wenden wir wieder den Algorithmus an:\\[1em]
(\Algo{R, U, Rp, U}) (\Algo{R, U, U, Rp})\\
\parbox{0.7\linewidth}{
	Danach erhalten wir das \textbf{Fisch} Muster, allerdings um 180° gedreht. Das entspricht wieder Fall 1 aus der Anfängermethode.
}\parbox{0.3\linewidth}{
	\centering
	\OLLU
	\RubikRotation{R, U, Rp, U, R, U, U, Rp}
	\RubikRotation{yp}
	\RubikRotation{R, U, Rp, U, R, U, U, Rp}
	\ShowCube{15mm}{0.35}{\DrawRubikFaceUpSide}
}\\[1em]
\parbox{0.7\linewidth}{
	Wir müssen also den Würfel so drehen, dass die korrekte Ecke links unten ist:
}\parbox{0.3\linewidth}{
	\centering
	\OLLU
	\RubikRotation{R, U, Rp, U, R, U, U, Rp}
	\RubikRotation{yp}
	\RubikRotation{R, U, Rp, U, R, U, U, Rp}
	\RubikRotation{y, y}
	\ShowCube{15mm}{0.35}{\DrawRubikFaceUpSide}
}\\[1em]
Jetzt wenden wir noch einmal den Algorithmus an:\\[1em]
(\Algo{R, U, Rp, U}) (\Algo{R, U, U, Rp})\\[1em]
\parbox{0.3\linewidth}{
	\OLLU
	\RubikRotation{R, U, Rp, U, R, U, U, Rp}
	\RubikRotation{yp}
	\RubikRotation{R, U, Rp, U, R, U, U, Rp}
	\RubikRotation{y, y}
	\RubikRotation{R, U, Rp, U, R, U, U, Rp}
	\ShowCube{15mm}{0.35}{\DrawRubikFaceUpSide}\\
}\parbox{0.7\linewidth}{
	Und jetzt ist die gelbe Fläche fertig.
}\\[3em]
Hier einmal ein Diagramm, das für alle acht Muster erläutert, wie man sie nach der Anfängermethode in die fertige gelbe Fläche überführt:
\begin{center}
	\begin{tikzpicture}[nodes={draw, thick, rounded corners, font=\bfseries}]
		\node[] (T) {T};
		\node[] (U) [above=of T] {U};
		\node[] (L) [below=of T] {L};

		\node[] (Antifisch) [right=of T] {Anti-Fisch};
		\node[] (Fisch) [right=of Antifisch] {Fisch};
		\node[] (Pi) [above=of Fisch] {Pi};
		\node[] (H) [below=of Fisch] {H};
		\node[] (Fertig) [right=of Fisch] {Fertig};

		\draw[->] (T) to [out=0, in=180] (Antifisch);
		\draw[->] (U) to [out=0, in=90] (Antifisch);
		\draw[->] (L) to [out=0, in=270] (Antifisch);
		\draw[->] (Antifisch) to (Fisch);
		\draw[->] (Pi) to [out=270, in=90] (Fisch);
		\draw[->] (H) to (Fisch);
		\draw[->] (Fisch) to (Fertig);
	\end{tikzpicture}
\end{center}

Analog zu dem Beispiel oben, zeigt das Diagramm die drei Schritte bis zur fertigen gelben Fläche. Man kann auch sehen, dass man auch für das \textbf{T} Muster und das \textbf{L} Muster drei Schritte braucht um zur gelben Fläche zu kommen. Und dass man zwei Schritte braucht, um vom \textbf{Anti-Fisch}, \textbf{Pi} Muster und \textbf{H} Muster zur fertigen gelben Fläche zu kommen. Allein das \textbf{Fisch} Muster kann man in einem Schritt lösen.

Wir können hier jetzt schneller werden, indem wir Für jedes der sechs übrigen Muster einen Algorithmus lernen, mit dem man in einem Schritt zur gelben Fläche gelangt.
Diese werden wir Dir jetzt vorstellen und dann kannst Du sie sukzessive lernen.
Hast Du für ein Muster den Algorithmus noch nicht auswendig gelernt, greif einfach auf die Anfängermethode zurückel!
Mit welchem Muster Du anfangen willst, das ist Dir überlassen.

Die folgenden Algorithmen haben eine interessante Eigenschaft, die man für das Auswendiglernen ausnutzen kann.
Wenn man bei einer gelben Fläche anfängt und einen Algorithmus zwei bis drei Mal wiederholt erhält man wieder die gelbe Fläche.
Du kannst also einen Algorithmen üben, indem Du bei einer gelben Fläche startest und den Algorithmus so lange wiederholst, bis Du wieder bei der gelben Fläche angekommen bist.
Das hilft Dir Dann auch, den Algorithmus in Deinem Muskelgedächtnis zu verankern und du wirst immer schneller.

\subsection{Muster und Algorithmen}
\parbox{0.3\linewidth}{
	\centering
	\textbf{Fisch} \\[1ex]
  \RubikCubeGreyAll%
  \RubikFaceUp{X}{Y}{X}
	      {Y}{Y}{Y}
	      {Y}{Y}{X}%
  \RubikFaceFront{X}{X}{Y}
		 {X}{X}{X}
		 {X}{X}{X}%
  \RubikFaceRight{X}{X}{Y}
		 {X}{X}{X}
		 {X}{X}{X}%
  \RubikFaceLeft{X}{X}{X}
		{X}{X}{X}
		{X}{X}{X}%
  \RubikFaceBack{X}{X}{Y}
		{X}{X}{X}
		{X}{X}{X}%
  %\ShowCube{2cm}{0.35}{\DrawRubikCubeRU}
  \ShowCube{\linewidth}{0.35}{%
    \DrawRubikFaceUpSide
  }
}%
\parbox{0.7\linewidth}{
	(\Algo{R, U, Rp, U}) \\[1em] (\Algo{R, U, U, Rp})
}\\[1em]
%
%
\parbox{0.3\linewidth}{
\begin{center}
	\textbf{Anti-Fisch} \\[1ex]
  \RubikCubeGreyAll%
  \RubikFaceUp{X}{Y}{Y}
	      {Y}{Y}{Y}
	      {X}{Y}{X}%
  \RubikFaceFront{Y}{X}{X}
		 {X}{X}{X}
		 {X}{X}{X}%
  \RubikFaceRight{Y}{X}{X}
		 {X}{X}{X}
		 {X}{X}{X}%
  \RubikFaceLeft{Y}{X}{X}
		{X}{X}{X}
		{X}{X}{X}%
  \RubikFaceBack{X}{X}{X}
		{X}{X}{X}
		{X}{X}{X}%
  %\ShowCube{2cm}{0.35}{\DrawRubikCubeRU}
  \ShowCube{1.5cm}{0.35}{%
    \DrawRubikFaceUpSide
  }
\end{center}
}%
\parbox{0.7\linewidth}{
	(\Algo{R, U, U, Rp, Up}) \\[1em] (\Algo{R, Up, Rp})
}\\[1em]
%
%
\parbox{0.3\linewidth}{
\begin{center}
  \textbf{H} \\
  \RubikCubeGreyAll%
  \RubikFaceUp{X}{Y}{X}
	      {Y}{Y}{Y}
	      {X}{Y}{X}%
  \RubikFaceFront{X}{X}{X}
		 {X}{X}{X}
		 {X}{X}{X}%
  \RubikFaceRight{Y}{X}{Y}
		 {X}{X}{X}
		 {X}{X}{X}%
  \RubikFaceLeft{Y}{X}{Y}
		{X}{X}{X}
		{X}{X}{X}%
  \RubikFaceBack{X}{X}{X}
		{X}{X}{X}
		{X}{X}{X}%
  %\ShowCube{2cm}{0.35}{\DrawRubikCubeRU}
  \ShowCube{1.5cm}{0.35}{%
    \DrawRubikFaceUpSide
  }
\end{center}
}%
\parbox{0.7\linewidth}{
	(\Algo{R, U, Rp, U}) \\[1em]
	(\Algo{R, Up, Rp, U}) \\[1em]
  (\Algo{R, U, U, Rp})
}\\[1em]
%
%
\parbox{0.3\linewidth}{
\begin{center}
  \textbf{L} \\
  \RubikCubeGreyAll%
  \RubikFaceUp{Y}{Y}{X}
	      {Y}{Y}{Y}
	      {X}{Y}{Y}%
  \RubikFaceFront{Y}{X}{X}
		 {X}{X}{X}
		 {X}{X}{X}%
  \RubikFaceRight{X}{X}{Y}
		 {X}{X}{X}
		 {X}{X}{X}%
  \RubikFaceLeft{X}{X}{X}
		{X}{X}{X}
		{X}{X}{X}%
  \RubikFaceBack{X}{X}{X}
		{X}{X}{X}
		{X}{X}{X}%
  %\ShowCube{2cm}{0.35}{\DrawRubikCubeRU}
  \ShowCube{1.5cm}{0.35}{%
    \DrawRubikFaceUpSide
  }
\end{center}
}%
\parbox{0.7\linewidth}{
	(\Algo{F, Rp, Fp, Rw}) \\[1em] (\Algo{U, R, Up, Rwp})
}\\[1em]
%
%
\parbox{0.3\linewidth}{
\begin{center}
  \textbf{Pi} \\
  \RubikCubeGreyAll%
  \RubikFaceUp{X}{Y}{X}
	      {Y}{Y}{Y}
	      {X}{Y}{X}%
  \RubikFaceFront{X}{X}{Y}
		 {X}{X}{X}
		 {X}{X}{X}%
  \RubikFaceRight{X}{X}{X}
		 {X}{X}{X}
		 {X}{X}{X}%
  \RubikFaceLeft{Y}{X}{Y}
		{X}{X}{X}
		{X}{X}{X}%
  \RubikFaceBack{Y}{X}{X}
		{X}{X}{X}
		{X}{X}{X}%
  %\ShowCube{2cm}{0.35}{\DrawRubikCubeRU}
  \ShowCube{1.5cm}{0.35}{%
    \DrawRubikFaceUpSide
  }
\end{center}
}%
\parbox{0.7\linewidth}{
	\Algo{R, U, U} \\[1em]
	(\Algo{R, R, Up}) (\Algo{R, R, Up}) \\[1em]
  \Algo{R, R, U, U, R}
}\\[1em]
%
%
\parbox{0.3\linewidth}{
\begin{center}
  \textbf{T} \\
  \RubikCubeGreyAll%
  \RubikFaceUp{X}{Y}{Y}
	      {Y}{Y}{Y}
	      {X}{Y}{Y}%
  \RubikFaceFront{Y}{X}{X}
		 {X}{X}{X}
		 {X}{X}{X}%
  \RubikFaceRight{X}{X}{X}
		 {X}{X}{X}
		 {X}{X}{X}%
  \RubikFaceLeft{X}{X}{X}
		{X}{X}{X}
		{X}{X}{X}%
  \RubikFaceBack{X}{X}{Y}
		{X}{X}{X}
		{X}{X}{X}%
  %\ShowCube{2cm}{0.35}{\DrawRubikCubeRU}
  \ShowCube{1.5cm}{0.35}{%
    \DrawRubikFaceUpSide
  }
\end{center}
}%
\parbox{0.7\linewidth}{
	(\Algo{Rw, U, Rp, Up}) \\[1em] (\Algo{Rwp, F, R, Fp})
}\\[1em]
%
%
\parbox{0.3\linewidth}{
\begin{center}
  \textbf{U} \\
  \RubikCubeGreyAll%
  \RubikFaceUp{Y}{Y}{Y}
	      {Y}{Y}{Y}
	      {X}{Y}{X}%
  \RubikFaceFront{Y}{X}{Y}
		 {X}{X}{X}
		 {X}{X}{X}%
  \RubikFaceRight{X}{X}{X}
		 {X}{X}{X}
		 {X}{X}{X}%
  \RubikFaceLeft{X}{X}{X}
		{X}{X}{X}
		{X}{X}{X}%
  \RubikFaceBack{X}{X}{X}
		{X}{X}{X}
		{X}{X}{X}%
  %\ShowCube{2cm}{0.35}{\DrawRubikCubeRU}
  \ShowCube{1.5cm}{0.35}{%
    \DrawRubikFaceUpSide
  }
\end{center}
}%
\parbox{0.7\linewidth}{
	\Algo{R, R, D} \\[1em] (\Algo{Rp, U, U, R}) \\[1em]
	\Algo{Dp} \\[1em]
	(\Algo{Rp, U, U, Rp})
}\\[1em]
