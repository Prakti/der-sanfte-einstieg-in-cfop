\selectlanguage{ngerman}
\section{Eine Abkürzung für das gelbe Kreuz}
Die vier Fälle auf dem Weg zum gelben Kreuz sind Dir jetzt bestimmt geläufig:\\[1em]
\RubikCubeGreyAll%
\RubikFaceUp{X}{X}{X}
            {X}{Y}{X}
            {X}{X}{X}%
\RubikFaceFront{X}{Y}{X}
               {R}{R}{R}
	       {R}{R}{R}%
\RubikFaceRight{X}{Y}{X}
	       {G}{G}{G}
	       {G}{G}{G}%
\ShowCube{2.5cm}{0.4}{%
  \DrawRubikCubeRU%
}
$\Longrightarrow$
\RubikCubeGreyAll%
\RubikFaceUp{X}{Y}{X}
            {Y}{Y}{X}
            {X}{X}{X}%
\RubikFaceFront{X}{Y}{X}
               {R}{R}{R}
	       {R}{R}{R}%
\RubikFaceRight{X}{Y}{X}
	       {G}{G}{G}
	       {G}{G}{G}%
\ShowCube{2.5cm}{0.4}{%
  \DrawRubikCubeRU%
}
$\Longrightarrow$
\RubikCubeGreyAll%
\RubikFaceUp{X}{X}{X}
            {Y}{Y}{Y}
            {X}{X}{X}%
\RubikFaceFront{X}{Y}{X}
               {R}{R}{R}
	       {R}{R}{R}%
\RubikFaceRight{X}{X}{X}
	       {G}{G}{G}
	       {G}{G}{G}%
\ShowCube{2.5cm}{0.4}{%
  \DrawRubikCubeRU%
}
$\Longrightarrow$
\RubikCubeGreyAll%
\RubikFaceUp{X}{Y}{X}
            {Y}{Y}{Y}
            {X}{Y}{X}%
\RubikFaceFront{X}{X}{X}
               {R}{R}{R}
	       {R}{R}{R}%
\RubikFaceRight{X}{X}{X}
	       {G}{G}{G}
	       {G}{G}{G}%
\ShowCube{2.5cm}{0.4}{%
  \DrawRubikCubeRU%
}\\[1em]

Wir können mit dem Folgenden Algorithmus von Fall 2 direkt zu Fall 4 kommen indem wir den Würfel wie folgt halten und dann folgenden Algorithmus anwenden:
\begin{center}
  \RubikCubeGreyAll%
  \RubikFaceUp{X}{X}{X}
	      {X}{Y}{Y}
	      {X}{Y}{X}%
  \RubikFaceFront{X}{X}{X}
		 {R}{R}{R}
		 {R}{R}{R}%
  \RubikFaceRight{X}{X}{X}
		 {G}{G}{G}
		 {G}{G}{G}%
  \ShowCube{2.5cm}{0.4}{%
    \DrawRubikCubeRU%
  }
  $\Longrightarrow$
  \RubikCubeGreyAll%
  \RubikFaceUp{X}{Y}{X}
	      {Y}{Y}{Y}
	      {X}{Y}{X}%
  \RubikFaceFront{X}{X}{X}
		 {R}{R}{R}
		 {R}{R}{R}%
  \RubikFaceRight{X}{X}{X}
		 {G}{G}{G}
		 {G}{G}{G}%
  \ShowCube{2.5cm}{0.4}{%
    \DrawRubikCubeRU%
  }\\[2em]
  \sffamily\Large (\Algo{Fw, R, U}) (\Algo{Rp, Up, Fwp})
\end{center}

Das ist quasi der selbe Algorithmus wie aus der Anfängermethode, nur dass die beiden Drehungen der Front breiter sind.
Damit kann man gleich Zeit sparen ohne sich viel merken zu müssen. Ist doch Super!
