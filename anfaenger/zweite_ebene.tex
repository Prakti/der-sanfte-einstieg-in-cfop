\section{Die zweite Ebene fertig stellen}
\parbox{0.7\linewidth}{
In diesem Schritt geht es darum, die richtigen Kantensteine in die mittlere Ebene zu bekommen.
Nach diesem Schritt hast Du die beiden ersten Ebenen von des Würfels gelöst.
Das sieht dann so aus wie rechts abgebildet:
}
\RubikCubeGreyAll%
\RubikFaceUp{X}{X}{X}
            {X}{Y}{X}
            {X}{X}{X}%
\RubikFaceFront{X}{X}{X}
               {R}{R}{R}
	       {R}{R}{R}%
\RubikFaceRight{X}{X}{X}
	       {G}{G}{G}
	       {G}{G}{G}%
\ShowCube{0.3\linewidth}{0.4}{\DrawRubikCubeRU}\\[1em]

Halte den Würfel so, dass die weiße Seite unten ist.
Suche dann einen Kantenstein auf der oberen Ebene, der keine gelbe Seite hat.
In diesem Beispiel haben wir einen Kantenstein mit den Farben Rot und Grün gefunden.

Wenn Du einen Kantenstein gefunden hast, der nicht gelb ist, drehe die obere Ebene so lange, bis du auf einer Seite des Würfels ein umgedrehtes „T" siehst.
Auf dem Bild oben ist es ein rotes T, aber es hätte auch ein blaues, orangefarbenes oder grünes sein können. 
Die Seite mit dem T sollte Dir zugewandt sein.

\subsection{Bewegen nach Rechts}
\RubikCubeGreyAll%
\RubikFaceUp{X}{X}{X}
            {X}{Y}{X}
            {X}{G}{X}%
\RubikFaceFront{X}{R}{X}
               {X}{R}{X}
	       {R}{R}{R}%
\RubikFaceRight{X}{X}{X}
	       {X}{G}{X}
	       {G}{G}{G}%
\ShowCube{0.3\linewidth}{0.4}{%
  \DrawRubikCubeRU%
  \draw[thick,->,color=black] (1.5,2.5) -- (2.5, 1.5);
}
\parbox{0.7\linewidth}{
  In diesem Fall soll die Kante nach \textbf{rechts} bewegt werden, da die andere
  Farbe des Kantensteins grün ist. Um den Kantenstein rechts von der mittleren
  Ebene nach unten zu bekommen, führe den folgenden Algorithmus aus:
}
\begin{center}
\Algo{U, R, Up, Rp, Up, Fp, U, F}.
\end{center}


\subsection{Bewegen nach Links}
Nachdem du ein T gebildet hast, wirst du feststellen, dass in einigen Fällen
das zu lösende Kantenstück auf die linke Seite muss, wie in der Abbildung oben
gezeigt.\\[1em]

\RubikCubeGreyAll%
\RubikFaceUp{X}{X}{X}
            {X}{Y}{X}
            {X}{B}{X}%
\RubikFaceFront{X}{R}{X}
               {X}{R}{X}
	       {R}{R}{R}%
\RubikFaceRight{X}{X}{X}
	       {X}{G}{X}
	       {G}{G}{G}%
\ShowCube{0.3\linewidth}{0.4}{%
  \DrawRubikCubeRU%
  \draw[thick,->,color=black] (1.5,2.5) -- (0.5, 1.5);
}
\parbox{0.7\linewidth}{
  In diesem Beispiel soll ein Kantenstein mit den Farben Rot und Blau links von
  der mittleren Ebene platziert werden. Um dieses Problem zu lösen, führe den
  Algorithmus in umgekehrter Reihenfolge aus, wie wir ihn in diesem Schritt
  vorgestellt haben. Der Algorithmus sieht dann folgendermaßen aus:
}
\begin{center}
\Algo{Up, Lp, U, L, U, F, Up, Fp}
\end{center}

\subsection{Sonderfälle}

\parbox{0.7\linewidth}{
  \paragraph{Situation 1}
  Der Kantenstein ist an der richtigen Position aber spiegelverkert ausgerichtet.\\[1em]
}
\RubikCubeGreyAll%
\RubikFaceUp{X}{X}{X}
            {X}{Y}{X}
            {X}{X}{X}%
\RubikFaceFront{X}{X}{X}
               {X}{R}{G}
	       {R}{R}{R}%
\RubikFaceRight{X}{X}{X}
	       {R}{G}{X}
	       {G}{G}{G}%
\ShowCube{0.3\linewidth}{0.4}{%
  \DrawRubikCubeRU%
}\\[1em]
%%
\parbox{0.7\linewidth}{
  \paragraph{Situation 2}
  Der Kantenstein ist an der falschen Stelle innerhalb der mittleren Ebene.\\[1em]
}
\RubikCubeGreyAll%
\RubikFaceUp{X}{X}{X}
            {X}{Y}{X}
            {X}{X}{X}%
\RubikFaceFront{X}{X}{X}
               {X}{O}{G}
	       {O}{O}{O}%
\RubikFaceRight{X}{X}{X}
	       {R}{G}{X}
	       {G}{G}{G}%
\ShowCube{0.3\linewidth}{0.4}{%
  \DrawRubikCubeRU%
}\\[1em]

Um diese Sonderfälle zu lösen, suche Dir zunächst einen Kantenstein mit einer
gelben Seite und bringe ihn so in Position als würdest Du ihn gerne nach links
oder nach rechts an die Stelle bringen wo jetzt der unpassende Kantenstein ist.
Führe den passenden Algorithmus aus, um den gelben Kantenstein an diese Position
zu bringen. Damit ist dann der andere Stein in der oberen Ebene und kann wie
oben dargestellt mit einem der beiden Algorithmen eingebaut werden.

Jetzt haben wir die ersten zwei Ebenen fertig und damit das "'F"' der CFOP Methode abgehakt.
