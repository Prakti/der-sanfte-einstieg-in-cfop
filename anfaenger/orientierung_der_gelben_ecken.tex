\section{Orientierung der gelben Ecken}
\parbox{0.7\linewidth}{
In diesem Schritt komplettieren wir das "'O"' von CFOP indem wir die Ecken um das Kreuz herum so orientieren, sodass sich eine gelbe Fläche ergibt.
Ob die anderen Farben stimmen ist jetzt noch nicht wichtig.
Darum kümmern wir uns später.

Folgende Fälle sind möglich:
}
\RubikCubeGreyAll%
\RubikFaceUp{Y}{Y}{Y}
            {Y}{Y}{Y}
            {Y}{Y}{Y}%
\RubikFaceFront{X}{X}{X}
               {R}{R}{R}
	       {R}{R}{R}%
\RubikFaceRight{X}{X}{X}
	       {G}{G}{G}
	       {G}{G}{G}%
\RubikFaceLeft{X}{X}{X}
	      {B}{B}{B}
	      {B}{B}{B}%
\RubikFaceBack{X}{X}{X}
	      {O}{O}{O}
	      {O}{O}{O}%
\ShowCube{0.3\linewidth}{0.4}{%
  \DrawRubikCubeRU
}\\[1em]

\paragraph{Fall 1}
\RubikCubeGreyAll%
\RubikFaceUp{Y}{Y}{Y}
            {Y}{Y}{Y}
            {Y}{Y}{Y}%
\RubikFaceFront{X}{X}{X}
               {R}{R}{R}
	       {R}{R}{R}%
\RubikFaceRight{X}{X}{X}
	       {G}{G}{G}
	       {G}{G}{G}%
\RubikFaceLeft{X}{X}{X}
	      {B}{B}{B}
	      {B}{B}{B}%
\RubikFaceBack{X}{X}{X}
	      {O}{O}{O}
	      {O}{O}{O}%
\ShowCube{2cm}{0.4}{\DrawRubikCubeRU}
\ShowCube{2.5cm}{0.4}{%
  \DrawRubikFaceUpSide
}\\[1em]
Alle Ecken sind bereits richtig orientiert. Hier sind wir dierekt fertig.

\paragraph{Fall 2}
\RubikCubeGreyAll%
\RubikFaceUp{X}{Y}{X}
            {Y}{Y}{Y}
            {Y}{Y}{X}%
\RubikFaceFront{X}{X}{X}
               {R}{R}{R}
	       {R}{R}{R}%
\RubikFaceRight{X}{X}{X}
	       {G}{G}{G}
	       {G}{G}{G}%
\RubikFaceLeft{X}{X}{X}
	      {B}{B}{B}
	      {B}{B}{B}%
\RubikFaceBack{X}{X}{X}
	      {O}{O}{O}
	      {O}{O}{O}%
\ShowCube{2cm}{0.4}{\DrawRubikCubeRU}
\ShowCube{2.5cm}{0.4}{%
  \DrawRubikFaceUpSide
}\\[1em]
Eine Ecke ist richtig orientiert. Hier dreht man den Würfel so, dass diese
Ecke vorne links liegt und führt dann den Algorithmus
\begin{center}
	\Algo{R, U, Rp, Up, R, U, U, Rp}
\end{center}
aus. Danach schaut man wieder welcher Fall nun vorliegt.

\paragraph{Fall 3}
Beispiel:
\RubikCubeGreyAll%
\RubikFaceUp{Y}{Y}{Y}
            {Y}{Y}{Y}
            {X}{Y}{X}%
\RubikFaceFront{Y}{X}{X}
               {R}{R}{R}
	       {R}{R}{R}%
\RubikFaceRight{X}{X}{X}
	       {G}{G}{G}
	       {G}{G}{G}%
\RubikFaceLeft{X}{X}{X}
	      {B}{B}{B}
	      {B}{B}{B}%
\RubikFaceBack{X}{X}{X}
	      {O}{O}{O}
	      {O}{O}{O}%
\ShowCube{2cm}{0.4}{\DrawRubikCubeRU}
\ShowCube{2.5cm}{0.4}{%
  \DrawRubikFaceUpSide
}\\[1em]
Genau Zwei Ecken sind richtig orientiert. Dabei ist es unwichtig welches Muster sie konkret ergeben.
In diesem Fall dreht man den Würfel so, dass eine falsche Ecke vorne links liegt und mit
der gelben Seite zu einem zeigt. Dann wendet man wieder den Algorithmus aus Fall 2 an:
\begin{center}
  \Algo{R, U, Rp, Up, R, U, U, Rp}
\end{center}
aus. Danach schaut man erneut welcher Fall nun vorliegt.

\paragraph{Fall 4}
Beispiel:
\RubikCubeGreyAll%
\RubikFaceUp{X}{Y}{X}
            {Y}{Y}{Y}
            {X}{Y}{X}%
\RubikFaceFront{X}{X}{X}
               {R}{R}{R}
	       {R}{R}{R}%
\RubikFaceRight{X}{X}{X}
	       {G}{G}{G}
	       {G}{G}{G}%
\RubikFaceLeft{X}{X}{Y}
	      {B}{B}{B}
	      {B}{B}{B}%
\RubikFaceBack{X}{X}{X}
	      {O}{O}{O}
	      {O}{O}{O}%
\ShowCube{2cm}{0.4}{\DrawRubikCubeRU}
\ShowCube{2.5cm}{0.4}{%
  \DrawRubikFaceUpSide
}\\[1em]
Keine Ecke ist richtig orientiert. Such Dir eine falsche Ecke und dreh den Würfel so,
dass sie vorne links liegt und mit ihrer gelben Seite nach links zeit. .
Wende wieder den Algorithmus aus Fall 2 an:
\begin{center}
  \Algo{R, U, Rp, Up, R, U, U, Rp}
\end{center}
aus. Danach schaut man erneut welcher Fall nun vorliegt.

Sobald wir in Fall 1 angekommen sind haben wir das "'O"' von CFOP fertig und
können in die letzte Phase gehen.
