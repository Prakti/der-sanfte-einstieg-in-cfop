\section{Das weiße Kreuz}
\parbox{0.7\linewidth}{
In nächsten Schritt soll ein weißes Kreuz gebildet werden, wie auf dem folgenden Bild zu sehen ist.
Um ein weißes Kreuz zu bilden, müssen die vier weißen Kantensteine um den weißen Mittelstein platziert werden.
}\parbox{0.3\linewidth}{
  \RubikCubeGreyAll%
  \RubikFaceUp{X}{W}{X}
	      {W}{W}{W}
	      {X}{W}{X}%
  \RubikFaceFront{X}{R}{X}
		 {X}{R}{X}
		 {X}{X}{X}%
  \RubikFaceRight{X}{B}{X}
		 {X}{B}{X}
		 {X}{X}{X}%
  \ShowCube{5cm}{0.4}{\DrawRubikCube}
}\\[1em]

Wie in der Illustration eben gezeigt, geht es dabei auch darum, dass die weißen Kantensteine des Kreuzes zu den Mittelsteinen der vier angrenzenden Flächen passen.
Also weiß-rot zu rot, weiß-blau zu blau, weiß-grün zu grün, weiß-orange zu orange.

Dies bewerkstelligen wir jetzt, indem wir die Oberseite des Würfels drehen, bis der weiße Kantenstein mit seiner anderen Farbe zum angrenzenden Mittelstein einer Seite passt.
Dann drehen wir diese Seite um 180°.
Im folgenden Beispiel zeigen wir das einmal für die grüne Seite. Erst drehen wir den grün-weißen Kantenstein über den grünen Eckstein, und dann drehen wir ihn "'runter"' auf die weiße Ebene:\\[1em]

\RubikCubeGreyAll%
\RubikFaceUp{X}{W}{X}
            {W}{Y}{W}
            {X}{W}{X}%
\RubikFaceFront{X}{G}{X}
               {X}{R}{X}
	       {X}{X}{X}%
\RubikFaceRight{X}{B}{X}
	       {X}{G}{X}
	       {X}{X}{X}%
\RubikFaceLeft{X}{O}{X}
	       {X}{B}{X}
	       {X}{X}{X}%
\ShowCube{2cm}{0.4}{\DrawRubikCube}
\quad\Rubik{Up}\RubikRotation{Up}
\ShowCube{2cm}{0.4}{\DrawRubikCube}
\quad\Rubik{R}\RubikRotation{R}
\ShowCube{2cm}{0.4}{\DrawRubikCube}
\quad\Rubik{R}\RubikRotation{R}
\ShowCube{2cm}{0.4}{\DrawRubikCube}\\[1em]

Damit hast Du den ersten Stein im weißen Kreuz gelöst.
Wiederhole dies für die anderen 3 Steine, um das komplette weiße Kreuz auf dem Würfel zu erzielen.
Am Ende soll der Würfel dann so aussehen:\\[1em]
\RubikCubeGreyAll%
\RubikFaceUp{X}{X}{X}
            {X}{Y}{X}
            {X}{X}{X}%
\RubikFaceFront{X}{X}{X}
               {X}{R}{X}
	       {X}{R}{X}%
\RubikFaceRight{X}{X}{X}
	       {X}{G}{X}
	       {X}{G}{X}%
\ShowCube{5cm}{0.4}{\DrawRubikCube}\\[1em]

Jetzt haben wir das "'C"' der CFOP Methode abgehakt und kümmern uns nacheinander um die
Fertigstellung der ersten und danach der zweiten Schicht.
