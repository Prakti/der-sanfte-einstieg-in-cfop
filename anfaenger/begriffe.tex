\section{Begriffe}
Damit Du alle Anleitungen und Hinweise in dieser Anleitung auch gut verstehen kannst, erklären wir Dir erst einmal ein paar Begriffe.
Dazu gehört die grundlegende Struktur des Würfels, wie die Seiten in dem Würfel benannt werden und wie man die weiter unten vorgestellten Algorithmen liest.

In der Speedcubing Szene ist ein Algorithmus eine Abfolge von Drehungen um ein bestimmtes Ergebnis zu erzielen.
Algorithmen bestehen oft aus vier bis neun Zügen.

Schau gerne nach jedem der folgenden Abschnitte einmal nach, ob Du das Vorgestellte am Würfel nachvollziehen kannst.

\subsection{Steine}

Ein Rubik's Würfel wird aus drei Arten von Steinen zusammen gesetzt. \\[1em]
\parbox{0.3\linewidth}{
  \centering
  \textbf{Mittelsteine}\\[1em] \RubikCubeGreyWY\ShowCube{2cm}{0.4}{\DrawRubikCube}
}
\parbox{0.7\linewidth}{
Das besondere an den Mittelsteinen ist, dass sie \emph{niemals} ihre Position verlassen können.
Im gelösten Zustand haben alle Flächen einer Seite die gleiche Farbe wie der Mittelstein.
Jeder der Seiten dreht sich um ihren Mittelstein.
}\\[1em]
\parbox{0.3\linewidth}{
  \centering
  \textbf{Kantensteine}\\[1em]
  \RubikCubeGreyAll%
  \RubikFaceUp{X}{W}{X}
	{W}{X}{W}
	{X}{W}{X}%
  \RubikFaceFront{X}{O}{X}
	   {O}{X}{O}
	   {X}{O}{X}%
  \RubikFaceRight{X}{G}{X}
	   {G}{X}{G}
	   {X}{G}{X}%
  \ShowCube{2cm}{0.4}{\DrawRubikCube}
}
\parbox{0.7\linewidth}{
  Die Kantensteine bestehen aus 2 Farben.
}\\[1em]
\parbox{0.3\linewidth}{
  \centering
  \textbf{Ecksteine}\\[1em]
  \RubikCubeGreyAll%
  \RubikFaceUp{W}{X}{W}
	{X}{X}{X}
	{W}{X}{W}%
  \RubikFaceFront{O}{X}{O}
	   {X}{X}{X}
	   {O}{X}{O}%
  \RubikFaceRight{G}{X}{G}
	   {X}{X}{X}
	   {G}{X}{G}%
  \ShowCube{2cm}{0.4}{\DrawRubikCube}
}
\parbox{0.7\linewidth}{
  Die Ecksteine bestehen aus 3 Farben.
}

\subsection{Rotationen}
Wie bereits in der Einleitung erwähnt, nennen wir die Techniken, mit denen wir jetzt schrittweise den Würfel lösen, Algorithmen.
Zu jedem der oben erwähnten vier bis acht Teilproblemen gibt es mindestens einen Algorithmus, um es zu lösen.

Jeder Algorithmus ist im Endeffekt eine Reihe von spezifischen Drehungen, welche wir mit Buchstaben beschreiben werden.

Wir werden jetzt alle für den Anfang notwendigen Drehungen und ihre Bezeichnung einmal vorstellen.
Damit Du ein besseres Bild vor Augen hast zeigen wir Dir hier einmal den teilweise aufgefalteten Würfel aus den Beispielen:\\[1em]

\RubikCubeSolved%
\ShowCube{\linewidth}{0.4}{\DrawRubikCubeSF}\\[1em]

Wie man sieht, ist die gelbe Farbe auf der Unterseite, die blaue links und die rote Farbe hinten!

\begin{instruction}{Rechte Seite im Uhrzeigersinn}
  \vspace{1.3em}
  \RubikCubeSolved%
  \ShowCube{15mm}{0.4}{\DrawRubikCube}%
  \quad\Rubik{R}\RubikRotation{R}
  \ShowCube{15mm}{0.4}{\DrawRubikCube}%
\end{instruction}
\hfil
\begin{instruction}{Rechte Seite gegen den Uhrzeigersinn}
  \RubikCubeSolved%
  \ShowCube{15mm}{0.4}{\DrawRubikCube}%
  \quad\Rubik{Rp}\RubikRotation{Rp}
  \ShowCube{15mm}{0.4}{\DrawRubikCube}%
\end{instruction}\\[3em]
\begin{instruction}{Linke Seite im Uhrzeigersinn}
  \vspace{1.3em}
  \RubikCubeSolved%
  \ShowCube{15mm}{0.4}{\DrawRubikCube}%
  \quad\Rubik{L}\RubikRotation{L}
  \ShowCube{15mm}{0.4}{\DrawRubikCube}%
\end{instruction}
\hfil
\begin{instruction}{Linke Seite gegen den Uhrzeigersinn}
  \RubikCubeSolved%
  \ShowCube{15mm}{0.4}{\DrawRubikCube}%
  \quad\Rubik{Lp}\RubikRotation{Lp}
  \ShowCube{15mm}{0.4}{\DrawRubikCube}%
\end{instruction}
\\[3em]
\begin{instruction}{Obere Seite im Uhrzeigersinn}
  \vspace{1.3em}
  \RubikCubeSolved%
  \ShowCube{15mm}{0.4}{\DrawRubikCube}%
  \quad\Rubik{U}\RubikRotation{U}
  \ShowCube{15mm}{0.4}{\DrawRubikCube}%
\end{instruction}
\hfil
\begin{instruction}{Obere Seite gegen den Uhrzeigersinn}
  \RubikCubeSolved%
  \ShowCube{15mm}{0.4}{\DrawRubikCube}%
  \quad\Rubik{Up}\RubikRotation{Up}
  \ShowCube{15mm}{0.4}{\DrawRubikCube}%
\end{instruction}
\\[3em]
\begin{instruction}{Vorne im Uhrzeigersinn}
  \RubikCubeSolved%
  \ShowCube{15mm}{0.4}{\DrawRubikCube}%
  \quad\Rubik{F}\RubikRotation{F}
  \ShowCube{15mm}{0.4}{\DrawRubikCube}%
\end{instruction}
\hfil
\begin{instruction}{Vorne gegen den Uhrzeigersinn}
  \RubikCubeSolved%
  \ShowCube{15mm}{0.4}{\DrawRubikCube}%
  \quad\Rubik{Fp}\RubikRotation{Fp}
  \ShowCube{15mm}{0.4}{\DrawRubikCube}%
\end{instruction}

Nimm Dir jetzt auch ruhig einmal Deinen Würfel zur Hand und probier ein wenig aus.
Das hilft Dir ein Gefühl für die Drehungen zu entwickeln und Dich später nicht zu verhaspeln.
