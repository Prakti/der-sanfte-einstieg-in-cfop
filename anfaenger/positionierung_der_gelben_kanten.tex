\section{Positionierung der gelben Kantensteine}
Dies ist die zweite Stufe des "'P"' Schrittes in CFOP.
Der Würfel müsste jetzt ungefähr so aussehen:
\begin{center}
  \RubikCubeGreyAll%
  \RubikFaceUp{Y}{Y}{Y}
	      {Y}{Y}{Y}
	      {Y}{Y}{Y}%
  \RubikFaceFront{R}{X}{R}
		 {R}{R}{R}
		 {R}{R}{R}%
  \RubikFaceRight{G}{X}{G}
		 {G}{G}{G}
		 {G}{G}{G}%
  \RubikFaceLeft{B}{X}{B}
		{B}{B}{B}
		{B}{B}{B}%
  \RubikFaceBack{O}{X}{O}
		{O}{O}{O}
		{O}{O}{O}%
  \ShowCube{2cm}{0.4}{\DrawRubikCubeRU}
  \ShowCube{2.5cm}{0.4}{\DrawRubikFaceUpSide}
\end{center}

Nur noch die gelben Kantensteine sind möglicherweise an der Falschen Position. Hier
haben wir wieder mehrere Fälle zu beachten:

\paragraph{Fall 1}
\RubikCubeGreyAll%
\RubikFaceUp{Y}{Y}{Y}
            {Y}{Y}{Y}
            {Y}{Y}{Y}%
\RubikFaceFront{R}{G}{R}
               {R}{R}{R}
	       {R}{R}{R}%
\RubikFaceRight{G}{B}{G}
	       {G}{G}{G}
	       {G}{G}{G}%
\RubikFaceLeft{B}{R}{B}
	      {B}{B}{B}
	      {B}{B}{B}%
\RubikFaceBack{O}{O}{O}
	      {O}{O}{O}
	      {O}{O}{O}%
\ShowCube{2cm}{0.4}{\DrawRubikCubeRU}
\ShowCube{2.5cm}{0.4}{%
  \DrawRubikFaceUpSide
}\\[1em]
Ein \emph{einziger} Kantenstein ist richtig. Wir halten jetzt den Würfel so, dass
die richtige Kante hinten liegt und wenden dann den folgenden Algorithmus an:
\begin{center}
	\sffamily\Large
	(\Algo{R, Up, R, U}) (\Algo{R, U, R, Up}) \\[1em] (\Algo{Rp, Up, R, R})
\end{center}

\paragraph{Fall 2}
\RubikCubeGreyAll%
\RubikFaceUp{Y}{Y}{Y}
            {Y}{Y}{Y}
            {Y}{Y}{Y}%
\RubikFaceFront{R}{G}{R}
               {R}{R}{R}
	       {R}{R}{R}%
\RubikFaceRight{G}{O}{G}
	       {G}{G}{G}
	       {G}{G}{G}%
\RubikFaceLeft{B}{R}{B}
	      {B}{B}{B}
	      {B}{B}{B}%
\RubikFaceBack{O}{B}{O}
	      {O}{O}{O}
	      {O}{O}{O}%
\ShowCube{2cm}{0.4}{\DrawRubikCubeRU}
\ShowCube{2.5cm}{0.4}{%
  \DrawRubikFaceUpSide
}\\[1em]
\emph{Kein} Kantenstein ist richtig. Hier wenden wir wieder den Algorithmus aus
Fall1 an und schauen dann erneut.
\begin{center}
	\sffamily\Large
	(\Algo{R, Up, R, U}) (\Algo{R, U, R, Up}) \\[1em] (\Algo{Rp, Up, R, R})
\end{center}

\paragraph{Fall 3}
\RubikCubeGreyAll%
\RubikFaceUp{Y}{Y}{Y}
            {Y}{Y}{Y}
            {Y}{Y}{Y}%
\RubikFaceFront{R}{R}{R}
               {R}{R}{R}
	       {R}{R}{R}%
\RubikFaceRight{G}{G}{G}
	       {G}{G}{G}
	       {G}{G}{G}%
\RubikFaceLeft{B}{B}{B}
	      {B}{B}{B}
	      {B}{B}{B}%
\RubikFaceBack{O}{O}{O}
	      {O}{O}{O}
	      {O}{O}{O}%
\ShowCube{2cm}{0.4}{\DrawRubikCubeRU}
\ShowCube{2.5cm}{0.4}{%
  \DrawRubikFaceUpSide
}\\[1em]
Alle vier Kanten sind richtig. Herzlichen Glückwunsch, der Würfel ist gelöst.
