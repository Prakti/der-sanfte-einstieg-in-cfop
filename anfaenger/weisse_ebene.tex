\section{Die weiße Ebene vervollständigen}
\parbox{0.7\linewidth}{
Wir vervollständigen jetzt die weiße Ebene, indem wir die Ecksteine um das Kreuz herum an ihre korrekte Position bringen.
Dabei werden wir auch darauf achten, das die Ecksteine richtig ausgerichtet sind.
Es soll nach diesem Schritt wie rechts abgebildet aussehen.

\emph{Aber:} Wir werden die weiße Ebene "'auf den Kopf gestellt"' lösen.
Das heißt: Du musst den Würfel so halten, dass die gelbe Seite oben ist.
}\parbox{0.3\linewidth}{
\RubikCubeGreyAll%
\RubikFaceUpAll{W}%
\RubikFaceFront{R}{R}{R}
               {X}{R}{X}
	       {X}{X}{X}%
\RubikFaceRight{B}{B}{B}
	       {X}{B}{X}
	       {X}{X}{X}%
\ShowCube{\linewidth}{0.4}{\DrawRubikCube}
}\\[1em]
Dieses "'auf dem Kopf gestellt lösen"' fühlt sich wahrscheinlich erst einmal umständlich an.
Aber probier es bitte aus.
Du wirst feststellen, dass Du auch durch ein leichtes Kippen des Würfels überprüfen kannst, ob Du gerade die weiße Ebene korrekt löst.

\enlargethispage{\baselineskip}
Zudem hat diese Vorgehensweise mehrere Vorteile:
Sie ist der Einstieg in eine sehr geläufige Bewegung, die in viele Algorithmen vorkommt.
Und sie trainiert jetzt schon Dein intuitives Gefühl, um später die ersten zwei Ebenen in einer Phase zu lösen.
Und dadurch dass Du jetzt alle Phasen mit der gelben Ebene nach oben ausführst, sparst Du Dir auch das Umdrehen des Würfels.
Das ist jetzt vielleicht nicht so gravierend, spart Dir später aber ein bis zwei wertvolle Sekunden.\\[1em]
\parbox{0.7\linewidth}{
Mit der gelben Seite nach oben, soll der Würfel nach dieser Phase dann so aussehen:

\emph{Hinweis:} die Balken unter dem Würfel zeigen die Farbe der verdeckten unteren Seite der vorderen Steine.
}\parbox{0.3\linewidth}{
\RubikCubeGreyAll%
\RubikFaceUp{X}{X}{X}
            {X}{Y}{X}
            {X}{X}{X}%
\RubikFaceFront{X}{X}{X}
               {X}{R}{X}
	       {R}{R}{R}%
\RubikFaceRight{X}{X}{X}
	       {X}{G}{X}
	       {G}{G}{G}%
\RubikFaceDown{W}{W}{W}
	      {W}{W}{W}
	      {W}{W}{W}%
\ShowCube{\linewidth}{0.4}{\DrawRubikCubeRU \DrawRubikCubeSidebarFD{RU}}
}

Um die Platzierung der Ecksteine zu bestimmen, schaue dir die zwei Mittelsteine um die zu lösende Ecke an:\\[1em]
\RubikCubeGreyAll%
\RubikFaceUp{X}{X}{X}
            {X}{Y}{X}
            {X}{X}{X}%
\RubikFaceFront{X}{X}{X}
               {X}{R}{X}
	       {X}{R}{X}%
\RubikFaceRight{X}{X}{X}
	       {X}{G}{X}
	       {X}{G}{X}%
\RubikFaceDown{X}{W}{X}
	      {W}{W}{W}
	      {X}{W}{X}%
\ShowCube{\linewidth}{0.4}{\DrawRubikCubeRU \DrawRubikCubeSidebarFD{RU}}\\[1em]
In diesem Beispiel werden die Ecksteine die Farben Rot, Grün und Weiß haben.

Bringe die Ecke über die Position, an die der Stein gelangen soll. Dazu drehst
Du einfach die oberste (gelbe) Ebene. Es sollte dann so aussehen wie in einem der 3
folgend dargestellten Fälle: \\[1em]

\RubikCubeGreyAll%
\RubikFaceUp{X}{X}{X}
            {X}{Y}{X}
            {X}{X}{G}%
\RubikFaceFront{X}{X}{R}
               {X}{R}{X}
	       {X}{R}{X}%
\RubikFaceRight{W}{X}{X}
	       {X}{G}{X}
	       {X}{G}{X}%
\RubikFaceDown{X}{W}{X}
	      {W}{W}{W}
	      {X}{W}{X}%
\ShowCube{0.3\linewidth}{0.4}{\DrawRubikCubeRU \DrawRubikCubeSidebarFD{RU}}
%
\RubikCubeGreyAll%
\RubikFaceUp{X}{X}{X}
            {X}{Y}{X}
            {X}{X}{W}%
\RubikFaceFront{X}{X}{G}
               {X}{R}{X}
	       {X}{R}{X}%
\RubikFaceRight{R}{X}{X}
	       {X}{G}{X}
	       {X}{G}{X}%
\RubikFaceDown{X}{W}{X}
	      {W}{W}{W}
	      {X}{W}{X}%
\ShowCube{0.3\linewidth}{0.4}{\DrawRubikCubeRU \DrawRubikCubeSidebarFD{RU}}
%
\RubikCubeGreyAll%
\RubikFaceUp{X}{X}{X}
            {X}{Y}{X}
            {X}{X}{R}%
\RubikFaceFront{X}{X}{W}
               {X}{R}{X}
	       {X}{R}{X}%
\RubikFaceRight{G}{X}{X}
	       {X}{G}{X}
	       {X}{G}{X}%
\RubikFaceDown{X}{W}{X}
	      {W}{W}{W}
	      {X}{W}{X}%
\ShowCube{0.3\linewidth}{0.4}{\DrawRubikCubeRU \DrawRubikCubeSidebarFD{RU}}
\\[3em]
Um diese Fälle zu lösen widerhole den Algorithmus
\begin{center}
  \Algo{R, U, Rp, Up}
\end{center}
bis die Ecke an der richtigen Stelle ist und die Farben auch richtig
ausgerichtet sind.

Gehe dann zu nächsten Ecke des weißen Kreuzes und bringe durch Drehen der
obersten Ebene wieder eine passende Ecke darüber und wiederhole den Algorithmus.

\subsection{Ein Sonderfall}
Es kann vorkommen, dass sich die zu lösende Ecke bereits auf der untersten Ebene befindet, aber nicht korrekt ausgerichtet (d.h. verdreht) ist.
Auch in diesem Fall wenden wir den oben beschriebenen Algorithmus an und wiederholen ihn, bis die Ecke korrekt gelöst ist:
\begin{center}
  \Algo{R, U, Rp, Up}
\end{center}
Hier zwei Beispiele für diesen Sonderfall:\\[1em]
\RubikCubeGreyAll%
\RubikFaceUp{X}{X}{X}
            {X}{Y}{X}
            {X}{X}{X}%
\RubikFaceFront{X}{X}{X}
               {X}{R}{X}
	       {X}{R}{W}%
\RubikFaceRight{X}{X}{X}
	       {X}{G}{X}
	       {R}{G}{X}%
\RubikFaceDown{X}{W}{G}
	      {W}{W}{W}
	      {X}{W}{X}%
\ShowCube{0.5\linewidth}{0.4}{\DrawRubikCubeRU \DrawRubikCubeSidebarFD{RU}}
%
\RubikCubeGreyAll%
\RubikFaceUp{X}{X}{X}
            {X}{Y}{X}
            {X}{X}{X}%
\RubikFaceFront{X}{X}{X}
               {X}{R}{X}
	       {X}{R}{G}%
\RubikFaceRight{X}{X}{X}
	       {X}{G}{X}
	       {W}{G}{X}%
\RubikFaceDown{X}{W}{R}
	      {W}{W}{W}
	      {X}{W}{X}%
\ShowCube{0.5\linewidth}{0.4}{\DrawRubikCubeRU \DrawRubikCubeSidebarFD{RU}}


\subsection{Noch ein Sonderfall}
Es kann vorkommen, dass eine Ecke sich auf der untersten (weißen) Ebene, aber an der falschen Ecke befindet:\\[1em]
\RubikCubeGreyAll%
\RubikFaceUp{X}{X}{X}
            {X}{Y}{X}
            {X}{X}{X}%
\RubikFaceFront{X}{X}{X}
               {X}{B}{X}
	       {X}{B}{G}%
\RubikFaceRight{X}{X}{X}
	       {X}{R}{X}
	       {W}{R}{X}%
\RubikFaceDown{X}{W}{R}
	      {W}{W}{W}
	      {X}{W}{X}%
\ShowCube{\linewidth}{0.4}{\DrawRubikCubeRU \DrawRubikCubeSidebarFD{RU}}

Dieses Problem lösen wir wieder mit dem oben bereits beschriebenen Algorithmus:
\begin{center}
\Algo{R, U, Rp, Up}
\end{center}
Diesmal wenden wir ihn einmal an, um den Ecksteine auf die oberste Ebene zu bekommen.
Dann verschieben wir ihn über seine Ziel-Ecke und wenden wieder den gewohnten Algorithmus an, um ihn dort korrekt zu platzieren.
