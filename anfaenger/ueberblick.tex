\section{Überblick über die Anfängermethode}
Um es zu Anfang einfacher zu haben, zerlegen wir die vier Phasen von CFOP jeweils nochmal in zwei Teilphasen.
Wir stellen Dir jetzt diese acht Teilschritte samt Ergebnist einmal kurz vor.

\emph{Ein Hinweis dabei:} bei Flächen die auf unseren Beispielen grau gefärbt sind ist uns die Farbe in dem Zwischenschritt egal.
Das ermöglichet es uns, Dir die wesentlichen Aspekte der jeweiligen Phase zu zeigen.

\paragraph{Cross (Kreuz):} \hfill \\[1em]
\parbox{0.7\linewidth}{
\textbf{Das Gänseblümchen:} Mit diesem Zwischenschritt macht man das Bilden des Kreuzes einfacher.
}\parbox{0.3\linewidth}{
\centering
\RubikCubeGreyAll%
\RubikFaceUp{X}{W}{X}
            {W}{Y}{W}
            {X}{W}{X}%
\RubikFaceFront{X}{X}{X}
	       {X}{O}{X}
	       {X}{X}{X}%
\RubikFaceRight{X}{X}{X}
	       {X}{G}{X}
	       {X}{X}{X}%
\ShowCube{3cm}{0.4}{\DrawRubikCube}
}\\[1em]
\parbox{0.7\linewidth}{
\textbf{Das weiße Kreuz:} Jetzt machen wir daraus das Kreuz auf der weißen Seite, wobei die Kantensteine der Farbe des Kreuzes mit den Mittelsteinen der angrenzenden Seiten ausgerichtet werden.
}\parbox{0.3\linewidth}{
\centering
\RubikCubeGreyAll%
\RubikFaceUp{X}{W}{X}
            {W}{W}{W}
            {X}{W}{X}%
\RubikFaceFront{X}{O}{X}
	       {X}{O}{X}
	       {X}{X}{X}%
\RubikFaceRight{X}{G}{X}
	       {X}{G}{X}
	       {X}{X}{X}%
\ShowCube{3cm}{0.4}{\DrawRubikCube}
}\\[1em]
\paragraph{F2L -- First 2 Layers (die ersten 2 Ebenen):}\hfill\\[1em]
\parbox{0.7\linewidth}{
\textbf{Die erste Ebene fertig stellen:} Nachdem das Kreuz fertig ist bringen wir die Ecksteine an ihre korrekte Stelle und richten sie aus.
}\parbox{0.3\linewidth}{
\centering
\RubikCubeGreyAll%
\RubikFaceUp{W}{W}{W}
            {W}{W}{W}
            {W}{W}{W}%
\RubikFaceFront{O}{O}{O}
	       {X}{O}{X}
	       {X}{X}{X}%
\RubikFaceRight{G}{G}{G}
	       {X}{G}{X}
	       {X}{X}{X}%
\ShowCube{3cm}{0.4}{\DrawRubikCube}
}\\[1em]
\parbox{0.7\linewidth}{
\textbf{Die zweite Ebene fertig stellen:} Jetzt lösen wir die zweite Ebene indem wir die passenden Kantensteine an ihren Platz bringen.
}\parbox{0.3\linewidth}{
\centering
\RubikCubeGreyAll%
\RubikFaceUp{W}{W}{W}
            {W}{W}{W}
            {W}{W}{W}%
\RubikFaceFront{O}{O}{O}
	       {O}{O}{O}
	       {X}{X}{X}%
\RubikFaceRight{G}{G}{G}
	       {G}{G}{G}
	       {X}{X}{X}%
\ShowCube{3cm}{0.4}{\DrawRubikCube}
}\\[1em]

\paragraph{OLL -- Orientation of the Last Layer (Ausrichtung der letzten Schicht):}\hfill\\[1em]
\parbox{0.7\linewidth}{
  \textbf{Das gelbe Kreuz:} Als Erstes erzeugen wir ein gelbes Kreuz in der letzten Ebene
}\parbox{0.3\linewidth}{
\centering
\RubikCubeGreyAll%
\RubikFaceUp{X}{Y}{X}
            {Y}{Y}{Y}
            {X}{Y}{X}%
\RubikFaceFront{X}{X}{X}
	       {G}{G}{G}
	       {G}{G}{G}%
\RubikFaceRight{X}{X}{X}
	       {O}{O}{O}
	       {O}{O}{O}%
\ShowCube{3cm}{0.4}{\DrawRubikCube}
}\\[1em]
\parbox{0.7\linewidth}{
  \textbf{Die gelbe Fläche:} Dann Vervollständigen wir die gelbe Fläche.
}\parbox{0.3\linewidth}{
\centering
\RubikCubeGreyAll%
\RubikFaceUp{Y}{Y}{Y}
            {Y}{Y}{Y}
            {Y}{Y}{Y}%
\RubikFaceFront{X}{X}{X}
	       {G}{G}{G}
	       {G}{G}{G}%
\RubikFaceRight{X}{X}{X}
	       {O}{O}{O}
	       {O}{O}{O}%
\ShowCube{3cm}{0.4}{\DrawRubikCube}
}\\[1em]
\paragraph{PLL -- Permutation of the Last Layer (Positionierung der letzten Schicht):}\hfill\\[1em]
\parbox{0.7\linewidth}{
  \textbf{Die Ecksteine an ihre Position bringen:} Hier sorgen wir erst für die korrekte Positionierung der Ecksteine der letzten Ebene.
}\parbox{0.3\linewidth}{
\centering
\RubikCubeGreyAll%
\RubikFaceUp{Y}{Y}{Y}
            {Y}{Y}{Y}
            {Y}{Y}{Y}%
\RubikFaceFront{G}{X}{G}
	       {G}{G}{G}
	       {G}{G}{G}%
\RubikFaceRight{O}{X}{O}
	       {O}{O}{O}
	       {O}{O}{O}%
\ShowCube{3cm}{0.4}{\DrawRubikCube}
}\\[1em]
\parbox{0.7\linewidth}{
  \textbf{Die Kantensteine an ihre Position bringen:} Dann sorgen wir erst für die korrekte Positionierung der Kantensteine der letzten Ebene und damit ist der Würfel dann fertig gelöst.
}\parbox{0.3\linewidth}{
\centering
\RubikCubeGreyAll%
\RubikFaceUp{Y}{Y}{Y}
            {Y}{Y}{Y}
            {Y}{Y}{Y}%
\RubikFaceFront{G}{G}{G}
	       {G}{G}{G}
	       {G}{G}{G}%
\RubikFaceRight{O}{O}{O}
	       {O}{O}{O}
	       {O}{O}{O}%
\ShowCube{3cm}{0.4}{\DrawRubikCube}
}\\[1em]


Jetzt da Du einen groben Überblick hast, erläutern wir Dir jede dieser acht Phasen im Detail:
