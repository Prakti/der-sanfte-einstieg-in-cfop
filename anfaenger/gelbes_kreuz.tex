\section{Das gelbe Kreuz}
\parbox{0.7\linewidth}{
  Als Nächstes geht es uns darum, die letzte, gelbe Ebene so auszurichten, dass alle gelben Flächen in die richtige Richtung zeigen. 
  Wie angekündigt machen wir das in zwei Schritten.
  Der erste Schritt hat das Ziel, ein gelbes Kreuz zu bilden.
  Mit anderen Worten: Alle vier gelben Kantensteine sollen mit ihrer gelben Seite nach oben zeigen.
}
\RubikCubeGreyAll%
\RubikFaceUp{X}{Y}{X}
            {Y}{Y}{Y}
            {X}{Y}{X}%
\RubikFaceFront{X}{X}{X}
               {R}{R}{R}
	       {R}{R}{R}%
\RubikFaceRight{X}{X}{X}
	       {G}{G}{G}
	       {G}{G}{G}%
\ShowCube{0.3\linewidth}{0.4}{%
  \DrawRubikCubeRU%
}\\[1em]

Sobald die ersten zwei Ebenen des Würfels gelöst sind, kann man auf einen der vier folgenden Fälle stoßen:

\RubikCubeGreyAll%
\RubikFaceUp{X}{X}{X}
            {X}{Y}{X}
            {X}{X}{X}%
\RubikFaceFront{X}{Y}{X}
               {R}{R}{R}
	       {R}{R}{R}%
\RubikFaceRight{X}{Y}{X}
	       {G}{G}{G}
	       {G}{G}{G}%
\ShowCube{2.5cm}{0.4}{%
  \DrawRubikCubeRU%
}
$\Longrightarrow$
\RubikCubeGreyAll%
\RubikFaceUp{X}{Y}{X}
            {Y}{Y}{X}
            {X}{X}{X}%
\RubikFaceFront{X}{Y}{X}
               {R}{R}{R}
	       {R}{R}{R}%
\RubikFaceRight{X}{Y}{X}
	       {G}{G}{G}
	       {G}{G}{G}%
\ShowCube{2.5cm}{0.4}{%
  \DrawRubikCubeRU%
}
$\Longrightarrow$
\RubikCubeGreyAll%
\RubikFaceUp{X}{X}{X}
            {Y}{Y}{Y}
            {X}{X}{X}%
\RubikFaceFront{X}{Y}{X}
               {R}{R}{R}
	       {R}{R}{R}%
\RubikFaceRight{X}{X}{X}
	       {G}{G}{G}
	       {G}{G}{G}%
\ShowCube{2.5cm}{0.4}{%
  \DrawRubikCubeRU%
}
$\Longrightarrow$
\RubikCubeGreyAll%
\RubikFaceUp{X}{Y}{X}
            {Y}{Y}{Y}
            {X}{Y}{X}%
\RubikFaceFront{X}{X}{X}
               {R}{R}{R}
	       {R}{R}{R}%
\RubikFaceRight{X}{X}{X}
	       {G}{G}{G}
	       {G}{G}{G}%
\ShowCube{2.5cm}{0.4}{%
  \DrawRubikCubeRU%
}\\[1em]

Der vierte Fall ist das gesuchte gelbe Kreuz.
Manchmal ist man ohne etwas tun zu müssen in dieser Situation und ist direkt fertig.
Für die anderen drei Fälle wendest Du den folgen Algorithmus an: \\[1em]
\begin{center}
	\sffamily\Large (\Algo{F, R, U}) (\Algo{Rp, Up, Fp})
\end{center}

Wie durch die Pfeile in der Abbildung gezeigt bringt dies den Würfel von einer der Situationen zur nächsten.
Zwischendurch musst Du den Würfel immer wieder so drehen, dass er passend zu Dir ausgerichtet ist. 
Die obige Abbildung zeigt Dir die korrekte Ausrichtung für jeden der drei Fälle.
