\selectlanguage{ngerman}
\section{Positionierung der gelben Ecksteine}
Jetzt sind wir in der "'P"' Phase von CFOP angekommen, und müssen die gelben Steine so auf der obersten Schicht hin und her "'schieben"', dass alle Farben ringsherum korrekt sind.

Zunächst etwas Begriffsklärung: wenn zwei Ecksteine der letzten Ebene auf einer Seite die gleiche Farbe haben, dann nennt man diese Form einen "'Scheinwerfer"'.
Welche Farbe diese beiden Seiten dabei haben ist egal.

Im folgenden Beispiel sieht man zwei rote "'Scheinwerfer"' die sich auch zufällig auf der roten Seite befinden:
\begin{center}
  \RubikCubeGreyAll%
  \RubikFaceUp{Y}{Y}{Y}
	      {Y}{Y}{Y}
	      {Y}{Y}{Y}%
  \RubikFaceFront{R}{B}{R}
		 {R}{R}{R}
		 {R}{R}{R}%
  \RubikFaceRight{G}{R}{O}
		 {G}{G}{G}
		 {G}{G}{G}%
  \RubikFaceLeft{O}{O}{B}
		{B}{B}{B}
		{B}{B}{B}%
  \RubikFaceBack{G}{G}{B}
		{O}{O}{O}
		{O}{O}{O}%
  \ShowCube{2cm}{0.4}{\DrawRubikCubeRU}
  \ShowCube{2.5cm}{0.4}{\DrawRubikFaceUpSide}
\end{center}

Mit dieser Form lassen sich die folgenden Fälle beshreiben:

\paragraph{Fall 1}
\RubikCubeGreyAll%
\RubikFaceUp{Y}{Y}{Y}
            {Y}{Y}{Y}
            {Y}{Y}{Y}%
\RubikFaceFront{B}{X}{B}
               {R}{R}{R}
	       {R}{R}{R}%
\RubikFaceRight{R}{X}{R}
	       {G}{G}{G}
	       {G}{G}{G}%
\RubikFaceLeft{O}{X}{O}
	      {B}{B}{B}
	      {B}{B}{B}%
\RubikFaceBack{G}{X}{G}
	      {O}{O}{O}
	      {O}{O}{O}%
\ShowCube{2cm}{0.4}{\DrawRubikCubeRU}
\ShowCube{2.5cm}{0.4}{%
  \DrawRubikFaceUpSide
}\\[1em]
Hier sind Scheinwerfer sind auf allen vier Seiten.
Wir drehen jetzt die obere Schicht so lange bis die Scheinwerfer mit den Farben der Seite übereinstimmen und gehen dann zum nächsten Schritt.

\paragraph{Fall 2}
\RubikCubeGreyAll%
\RubikFaceUp{Y}{Y}{Y}
            {Y}{Y}{Y}
            {Y}{Y}{Y}%
\RubikFaceFront{X}{X}{X}
               {R}{R}{R}
	       {R}{R}{R}%
\RubikFaceRight{X}{X}{X}
	       {G}{G}{G}
	       {G}{G}{G}%
\RubikFaceLeft{R}{X}{R}
	      {B}{B}{B}
	      {B}{B}{B}%
\RubikFaceBack{X}{X}{X}
	      {O}{O}{O}
	      {O}{O}{O}%
\ShowCube{2cm}{0.4}{\DrawRubikCubeRU}
\ShowCube{2.5cm}{0.4}{%
  \DrawRubikFaceUpSide
}\\[1em]
Scheinwerfer sind auf \emph{einer einzigen} Seite. 
Hier drehen wir die Scheinwerfer zunächst so, dass sie nach links zeigen und wenden dann den folgenden Algorithmus an:
\begin{center}
	\sffamily\Large
	(\Algo{R, U, Rp, Up}) (\Algo{Rp, F, R}) \\[1em]
	(\Algo{R, Up, Rp, Up}) (\Algo{R, U, Rp, Fp})
\end{center}

\pagebreak
\paragraph{Fall 3}
\RubikCubeGreyAll%
\RubikFaceUp{Y}{Y}{Y}
            {Y}{Y}{Y}
            {Y}{Y}{Y}%
\RubikFaceFront{X}{X}{X}
               {R}{R}{R}
	       {R}{R}{R}%
\RubikFaceRight{X}{X}{X}
	       {G}{G}{G}
	       {G}{G}{G}%
\RubikFaceLeft{X}{X}{X}
	      {B}{B}{B}
	      {B}{B}{B}%
\RubikFaceBack{X}{X}{X}
	      {O}{O}{O}
	      {O}{O}{O}%
\ShowCube{2cm}{0.4}{\DrawRubikCubeRU}
\ShowCube{2.5cm}{0.4}{%
  \DrawRubikFaceUpSide
}\\[1em]
Es sind gar keine Scheinwerfer vorhanden. Wir wenden den selben Algorithmus
wie in Fall 2 an und schauen erneut:
\begin{center}
	\sffamily\Large
	(\Algo{R, U, Rp, Up}) (\Algo{Rp, F, R}) \\[1em]
	(\Algo{R, Up, Rp, Up}) (\Algo{R, U, Rp, Fp})
\end{center}
