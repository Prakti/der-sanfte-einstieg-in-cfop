\section{Positionierung der gelben Ecksteine}
Jetzt sind wir in der "'P"' Phase von CFOP angekommen, und müssen wir noch die gelben Steine so auf der obersten Schicht hin und
her "'schieben"', dass alle Farben ringsherum korrekt sind.
Als Erstes kümmern wir uns um die Ecksteine und dafür hält man zunächst
Ausschau nach sogenannten "'Scheinwerfern"', d.h. nach zwei Ecken,
bei denen die Farben auf einer Seitenfläche übereinstimmen. Die Farben der
Kanten sind dabei zunächst egal.

Im folgenden Beispiel sieht man zwei rote "'Scheinwerfer"' die sich auch
zufällig auf der roten Seite befinden:

\begin{center}
  \RubikCubeGreyAll%
  \RubikFaceUp{Y}{Y}{Y}
	      {Y}{Y}{Y}
	      {Y}{Y}{Y}%
  \RubikFaceFront{R}{B}{R}
		 {R}{R}{R}
		 {R}{R}{R}%
  \RubikFaceRight{G}{R}{O}
		 {G}{G}{G}
		 {G}{G}{G}%
  \RubikFaceLeft{O}{O}{B}
		{B}{B}{B}
		{B}{B}{B}%
  \RubikFaceBack{G}{G}{B}
		{O}{O}{O}
		{O}{O}{O}%
  \ShowCube{2cm}{0.4}{\DrawRubikCubeRU}
  \ShowCube{2.5cm}{0.4}{\DrawRubikFaceUpSide}
\end{center}

Damit ergeben sich für uns die folgenden Fälle:

\paragraph{Fall 1}
\RubikCubeGreyAll%
\RubikFaceUp{Y}{Y}{Y}
            {Y}{Y}{Y}
            {Y}{Y}{Y}%
\RubikFaceFront{B}{X}{B}
               {R}{R}{R}
	       {R}{R}{R}%
\RubikFaceRight{R}{X}{R}
	       {G}{G}{G}
	       {G}{G}{G}%
\RubikFaceLeft{O}{X}{O}
	      {B}{B}{B}
	      {B}{B}{B}%
\RubikFaceBack{G}{X}{G}
	      {O}{O}{O}
	      {O}{O}{O}%
\ShowCube{2cm}{0.4}{\DrawRubikCubeRU}
\ShowCube{2.5cm}{0.4}{%
  \DrawRubikFaceUpSide
}\\[1em]
Scheinwerfer sind auf allen vier Seiten. Hier drehen wir jetzt die obere
Schicht so lange bis die Scheinwerfer mit den Farben der Seite übereinstimmen
und gehen dann zum nächsten Schritt.

\paragraph{Fall 2}
\RubikCubeGreyAll%
\RubikFaceUp{Y}{Y}{Y}
            {Y}{Y}{Y}
            {Y}{Y}{Y}%
\RubikFaceFront{X}{X}{X}
               {R}{R}{R}
	       {R}{R}{R}%
\RubikFaceRight{R}{X}{R}
	       {G}{G}{G}
	       {G}{G}{G}%
\RubikFaceLeft{X}{X}{X}
	      {B}{B}{B}
	      {B}{B}{B}%
\RubikFaceBack{X}{X}{X}
	      {O}{O}{O}
	      {O}{O}{O}%
\ShowCube{2cm}{0.4}{\DrawRubikCubeRU}
\ShowCube{2.5cm}{0.4}{%
  \DrawRubikFaceUpSide
}\\[1em]
Scheinwerfer sind auf \emph{einer einzigen} Seite. Hier drehen wir die
Scheinwerfer zunächst so, dass sie nach rechts zeigen und wenden dann den
folgenden Algorithmus an:
\begin{center}
  \Algo{Fp, L, Fp, R, R, F, Lp, Fp, R, R, F, F}
\end{center}

\paragraph{Fall 3}
\RubikCubeGreyAll%
\RubikFaceUp{Y}{Y}{Y}
            {Y}{Y}{Y}
            {Y}{Y}{Y}%
\RubikFaceFront{X}{X}{X}
               {R}{R}{R}
	       {R}{R}{R}%
\RubikFaceRight{X}{X}{X}
	       {G}{G}{G}
	       {G}{G}{G}%
\RubikFaceLeft{X}{X}{X}
	      {B}{B}{B}
	      {B}{B}{B}%
\RubikFaceBack{X}{X}{X}
	      {O}{O}{O}
	      {O}{O}{O}%
\ShowCube{2cm}{0.4}{\DrawRubikCubeRU}
\ShowCube{2.5cm}{0.4}{%
  \DrawRubikFaceUpSide
}\\[1em]
Es sind gar keine Scheinwerfer vorhanden. Wir wenden den selben Algorithmus
wie in Fall 2 an und schauen erneut:
\begin{center}
  \Algo{Fp, L, Fp, R, R, F, Lp, Fp, R, R, F, F}
\end{center}
