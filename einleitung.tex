Diese Anleitung zielt darauf ab, es jedem Anfänger zu ermöglichen überhaupt einen Rubik's Würfel zu lösen.
Dabei wollen wir aber auch gleichzeitig die bestmöglichen Grundlagen legen, um später immer schneller darin zu werden.
Deshalb orientieren wir uns direkt an den Schritten und Zwischenergebnissen einer beliebten Speedcubing Methode: CFOP.

Damit Du Dir am Anfang nur ganz wenige Algorithmen merken musst, werden wir mit einer stark abgespeckten Variante dieser Methode anfangen.

Mit zunehmender Sicherheit wirst Du dann anfangen schneller zu werden und an die Grenzen der Anfängermethode stoßen.
Mit der Übung werden Dir Abläufe als Umständlich erscheinen und Du wirst Dir wünschen, mehrere Schritte in einem zu erledigen.
Hier wollen wir Dir mit diesem Leitfaden die Möglichkeite geben, Stück für Stück neue Algorithmen dazu zu nehmen, bis Du CFOP vollständig beherrschst.

Die CFOP-Methode basiert auf der Idee, dass sich jedes Problem einfach und wiederholbar lösen lässt, indem man es in Teilprobleme zerlegt, welche man dann einzeln löst.
Der Begriff CFOP leitet sich von den vier Teilprobleme ab in die das gesamte Lösen des Würfels zerlegt wird.
Es ist ein Akronym für Cross, F2L (First 2 Layers), OLL (Orientation of the Last Layer) und PLL (Permutation of the Last Layer):\\[1em]
\parbox{0.7\linewidth}{
\textbf{Cross (Kreuz):} Zuerst wird ein Kreuz auf einer der Würfelseiten erstellt, wobei die Kantensteine der Farbe des Kreuzes mit den Mittelsteinen der angrenzenden Seiten ausgerichtet werden.
}\parbox{0.3\linewidth}{
\centering
\RubikCubeGreyAll%
\RubikFaceUp{X}{W}{X}
            {W}{W}{W}
            {X}{W}{X}%
\RubikFaceFront{X}{O}{X}
	       {X}{O}{X}
	       {X}{X}{X}%
\RubikFaceRight{X}{G}{X}
	       {X}{G}{X}
	       {X}{X}{X}%
\ShowCube{2cm}{0.4}{\DrawRubikCube}
}\\[1em]
\parbox{0.7\linewidth}{
\textbf{F2L -- First 2 Layers (die ersten 2 Ebenen):}
In diesem Schritt werden die ersten zwei Ebenen des Würfels gleichzeitig gelöst, indem Ecken und Kanten gepaart und an ihren Platz gebracht werden.
}\parbox{0.3\linewidth}{
\centering
\RubikCubeGreyAll%
\RubikFaceUp{W}{W}{W}
            {W}{W}{W}
            {W}{W}{W}%
\RubikFaceFront{O}{O}{O}
	       {O}{O}{O}
	       {X}{X}{X}%
\RubikFaceRight{G}{G}{G}
	       {G}{G}{G}
	       {X}{X}{X}%
\ShowCube{2cm}{0.4}{\DrawRubikCube}
}\\[1em]
\parbox{0.7\linewidth}{
  \textbf{OLL -- Orientation of the Last Layer (Ausrichtung der letzten Schicht):}
  Hier werden alle Steine der letzten Ebene so gedreht, dass die Oberseite eine einheitliche Farbe hat, ohne dabei die bereits gelösten Ebenen zu stören.
}\parbox{0.3\linewidth}{
\centering
\RubikCubeGreyAll%
\RubikFaceUp{Y}{Y}{Y}
            {Y}{Y}{Y}
            {Y}{Y}{Y}%
\RubikFaceFront{X}{X}{X}
	       {G}{G}{G}
	       {G}{G}{G}%
\RubikFaceRight{X}{X}{X}
	       {O}{O}{O}
	       {O}{O}{O}%
\ShowCube{2cm}{0.4}{\DrawRubikCube}
}\\[1em]
\parbox{0.7\linewidth}{
  \textbf{PLL -- Permutation of the Last Layer (Positionierung der letzten Schicht):} 
  Im letzten Schritt werden die Steine der letzten Ebene in ihre endgültige Position gebracht, wodurch der Würfel vollständig gelöst wird.
}\parbox{0.3\linewidth}{
\centering
\RubikCubeGreyAll%
\RubikFaceUp{Y}{Y}{Y}
            {Y}{Y}{Y}
            {Y}{Y}{Y}%
\RubikFaceFront{G}{G}{G}
	       {G}{G}{G}
	       {G}{G}{G}%
\RubikFaceRight{O}{O}{O}
	       {O}{O}{O}
	       {O}{O}{O}%
\ShowCube{2cm}{0.4}{\DrawRubikCube}
}\\[1em]

Für den Anfang werden wir jedes dieser Teilprobleme noch zusätzlich zerlegen, sodass wir den Würfel in insgesamt acht Phasen lösen.
Das ermöglicht es uns, mit nur einem kleinen Satz an Algorithmen für die Lösung der acht Teilprobleme auszukommen.

Ich spreche hier immer von 'wir' obwohl ich das Buch gerade alleine schreibe.
Das liegt daran, dass ich gerade Ideen und Algorithmen von ganz vielen Menschen aufnehme und entlang meiner eigenen Lernerfahrungen aufschreibe.

Viele Personen waren an der Entwicklung von CFOP beteiligt.
Vollständig dokumentiert wurde die Methode als erstes von Anneke Treep und Kurt Dockhorn im Jahr 1981.
Weiter systematisiert und popularisiert wurde CFOP von der tschechischen Speedcuberin Jessica Fridrich.
Bis heute benutzen viele berühmte Speedcuber wie Feliks Zemdegs oder Max Park diese Methoden, bzw. eine Weiterentwicklung davon.
